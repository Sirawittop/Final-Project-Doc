\setcounter{secnumdepth}{4}
\numberwithin{equation}{chapter}
\numberwithin{equation}{section}
\baselineskip=8mm
\chapter{บทนำ}

% \renewcommand{\thesubsection}{\thechapter.\arabic{subsection}}
\renewcommand{\thesubsection}{\arabic{subsection}.}
\renewcommand{\theequation}{\thesection\arabic{equation}}
\renewcommand{\thesubsubsection}{\thesubsection\arabic{subsubsection}}
\renewcommand{\thesection}{}
\renewcommand{\paragraph}{}
\section{ที่มาและความสำคัญของปัญหา}

เนื่องจากสถานการณ์การแพร่ระบาดของโรคติดเชื้อไวรัสโคโรนา 2019 (Covid -19) จากเมืองอู่ฮั่น(Wuhan) มณฑลหูเป่ย(Hubei) ประเทศสาธารณรัฐประชาชนจีน ทำให้มีการแพร่ระบาดขยายเป็นวงกว้างอย่างรวดเร็วไปยังประเทศต่างๆทั่วโลกรวมถึงประเทศไทยด้วย ทำให้ส่งผลกระทบต่อสาธารณสุข เศรษฐกิจ สังคม \cite{wuhan:hupei}
พยาบาลเป็นหนึ่งในทีมบุคลากรทางการแพทย์ที่มีบทบาทเป็นด่านหน้าในการควบคุมและป้องกันการแพร่ระบาดของโรค เป็นผู้ปฏิบัติงานโดยตรงกับผู้ป่วยต้องเข้าไปสัมผัสใกล้ชิดกับผู้ป่วย พยาบาลต้องตระหนักและดูแลป้องกันตนเองไม่ให้ติดเชื้อต้องมาตรฐานอย่างเคร่งครัด \cite{nurse:covid}
ทำให้พยาบาลเกิดความเหนื่อย ความเครียดจากภาระงานที่เพิ่มมากขึ้น ในการปฏิบัติงานพยาบาลต้องดูแลผู้ป่วยใน Home Isolation ที่ต้องรับคำปรึกษาตลอดเวลาและขึ้นเวร Community Isolation โรงพยาบาลสนามโดยไม่ได้หยุดพัก \cite{nurse:Isolation}
เมื่อเปรียบเทียบจำนวนพยาบาลกับสัดส่วนประชากรโดยมากถึง 1 : 353 \cite{people:nurse} ความเหนื่อยล้าจากการทํางานของพยาบาลส่งผลกระทบต่อ ความปลอดภัยของผู้ป่วยที่ลดลง การดูแลเอาใจใส่ผู้ป่วยที่อาจไม่ดีเท่าที่ควร พยาบาลต้องทํางานต่อเนื่องกันยาวนานถึง 12 ชั่วโมงอาจทําให้เกิดอัตราความผิดพลาดจากการทํางานเพิ่มขึ้น \cite{ngeaun:nurse}
โดยปกติแนวทางในการจัดตารางเวรของพยาบาลจะยึด แนวทางการบริหารการจัดตารางเวรหรือผลัด การเบิกเงินค่าตอบแทนนอกเวลา และค่าเวรหรือผลัด ของพยาบาลวิชาชีพ พยาบาลเทคนิค ผู้ช่วยพยาบาล กระทรวงสาธารณสุข เป็นหลักแต่สามารถปรับแต่งการจัดจัดตารางเวรให้เหมาะสมได้ \cite{tarang:nurse}
โดยการจัดตารางเวรหัวหน้าพยาบาลจะเป็นผู้จัดทำ การจัดตารางเวรแบบเดิมจะใช้การจดบันทึกในการดาษและเนื่องจากปัจจุบันเทคโนโลยีได้เข้ามามีบทบาทอย่างมากจึงเกิดการประยุกต์ใช้เทคโนโลยีเพื่อจัดตารางเวรของพยาบาล อาทิเช่นแอปพลิเคชัน “Microsoft Excel” ที่มาช่วยในการจัดตารางให้ดูง่ายสามารถคำนวณวันเวลาได้ลดความยุ่งยากในการจัดเก็บเอกสาร \cite{excel:headnurse}

จากการจัดตารางเวรข้างต้นของหัวหน้าพยาบาลแสดงให้เห็นว่าการจัดตารางเวรแบบเดิมหรือการจัดตารางเวรโดยการใช้แอปพลิเคชัน “Microsoft Excel” ก็ยังคงเกิดปัญหาในหลายๆเรื่อง เช่น การจัดตารางเวรจัดไม่เท่ากัน โดยอาจเกิดการเองเอียง ความเหลื่อมล้ำ โดยหัวหน้าพยาบาลไม่ได้มีข้อมูลในการชี้แจงที่ชัดเจน การจัดตารางเวรอาจไม่ได้ตรงตามความต้องการของพยาบาล ทำให้พยาบาลอาจมีการแลกเวรจำนวนมากๆ ซึ่งส่งผลให้การจัดตารางเวรนั้นเปลี่ยนไม่ตรงตามวัตถุประสงค์ที่หัวหน้าพยาบาลต้องการตั้งแต่ตั้น ข้อมูลตารางเวรหากมีการเปลี่ยนแปลงโดยการแลกเวรจะต้องทำการอัปเดตซึ่งเป็นไปได้ยาก 

จากปัญหาข้างต้นเพื่อแก้ไขปัญหาดังกล่าว ผู้จัดทำมุ่งเน้นการพัฒนาระบบผ่านเว็บแอปพลิเคชันเพื่อให้พยาบาลสามารถเข้าถึงและจัดการตารางเวรได้อย่างง่ายดายผ่านอินเทอร์เน็ต ระบบนี้จะช่วยลดความยุ่งยากในการจัดตารางเวร ตอบสนองความต้องการของพยาบาลในการจัดตารางเวร อำนวยความสะดวกในการขอลาและการแลกเวร โดยสามารถอัปเดตข้อมูลตารางเวรได้ง่ายขึ้น โดยหัวหน้าพยาบาลจะมีข้อมูลเก่าในระบบที่สามารถนำมาวิเคราะห์เพื่อจัดตารางเวรหรือตอบคำถามให้พยาบาลได้ ซึ่งทำให้ลดการถกเถียงในการจัดตารางเวร และมีภาพรวมให้ผู้อำนวยการโรงพยาบาลได้ทราบก่อนที่จะอนุมัติตารางเวรได้จึงทำให้โรงพยาบาลมีระบบระเบียบ มีประสิทธิภาพมากยิ่งขึ้น

\section{{วัตถุประสงค์}}

\hspace{0cm}\subsection{เพื่อสร้างแอปพลิเคชันอำนวยความสะดวกให้กับหัวหน้าพยาบาลและให้พยาบาล}

\hspace{0cm}\subsection{เพื่อเป็นการศึกษาและพัฒนาเว็บแอปพลิเคชัน}

\section{แนวคิดและหลักการ}

แอปพลิเคชันจัดตารางเวรพยาบาลถูกออกแบบมาเพื่อศึกษาและหาวิธีแก้ปัญหาเกี่ยวกับการจัดการตารางเวรของพยาบาล เช่น การจัดตารางเวร การแลกเวร การลา และปัญหาอื่นๆที่เกี่ยวข้อง แอปพลิเคชันนี้ถูกออกแบบมาเป็นเว็บแอปพลิเคชันเพื่อความสะดวกในการใช้งาน ผู้ใช้สามารถเข้าถึงแอปพลิเคชันได้ง่ายผ่านอินเทอร์เน็ตโดยไม่ต้องติดตั้งโปรแกรมใดๆเพิ่มเติม ในการออกแบบระบบได้นำหลักการวิเคราะห์และออกแบบระบบมาใช้โดยใช้ทฤษฎีวงจรชีวิตการพัฒนาซอฟต์แวร์ (Software Development Lifecycle : SDLC)
หน้าตาส่วนต่อประสานกับผู้ใช้ (UI) ได้รับการออกแบบโดยคำนึงถึงความสะดวกในการใช้งาน ใช้งานง่าย เข้าใจง่าย สวยงาม และสอดคล้องกับหลักการออกแบบ UI ทั่วไป โครงสร้างของระบบถูกออกแบบโดยใช้แผนภาพยูเอ็มแอล (Unified Modeling Language : UML) ซึ่งเป็นเครื่องมือที่ใช้ในการวิเคราะห์ ออกแบบ และสร้างแบบจำลองระบบ ช่วยให้เข้าใจโครงสร้างของระบบได้ง่าย และสามารถนำไปพัฒนาต่อได้สะดวก จากการออกแบบแอปพลิเคชันจัดตารางเวรพยาบาล คาดว่าจะช่วยแก้ปัญหาเกี่ยวกับการจัดการตารางเวรของพยาบาล เพิ่มประสิทธิภาพการทำงาน และช่วยให้พยาบาลสามารถจัดการตารางเวรของตัวเองได้สะดวกยิ่งขึ้น
\clearpage

\section{ขอบเขตการศึกษา}
\setcounter{secnumdepth}{4}

ผู้จัดทำแอปพลิเคชันจัดตารางเวรพยาบาลได้ทำการเก็บข้อมูลและได้ทำการออกแบบฟังก์ชันตามระดับของผู้ใช้งานโดยมีขอบเขตการทำงานโดยแบ่งผู้ใช้ออกเป็น 5 ประเภทดังนี้


\hspace{0cm}\textbf{\subsection{\textbf{ผู้ดูแลระบบแอปพลิเคชัน}}}

\hspace{1cm}\subsubsection{จัดการข้อมูลการเข้าสู่ระบบ}

\hspace{2.5cm}\paragraph{1.1.1 สามารถเข้าสู่ระบบ}

\hspace{2.5cm}\paragraph{1.1.2 สามารถแก้ไข Username และ Password}

\hspace{2.5cm}\paragraph{1.1.3 สามารถเพิ่ม ลบ ผู้ใช้งานในระบบ}

\hspace{2.5cm}\paragraph{1.1.4 สามารถออกจากระบบ}

\hspace{1cm}\subsubsection{จัดการโรงพยาบาล}

\hspace{2.5cm}\paragraph{1.2.1 สามารถเพิ่ม ลบ แก้ไข ข้อมูลเทียบเท่าผู้ดูแลระบบของโรงพยาบาล}

\hspace{1cm}\subsubsection{ดูข้อมูลสถิติการใช้งาน}

\hspace{2.5cm}\hangindent=4.9cm\hangafter=1\paragraph{1.3.1 สามารถดูข้อมูลการใช้งานของแต่ละโรงพยาบาล เช่น จำนวนการแลกเวรเฉลี่ยของพยาบาล จำนวนพยาบาลต่อวอร์ดโดยเฉลี่ย เป็นต้น}


\hspace{1cm}\subsubsection{จัดการสิทธิการใช้งาน}

\hspace{2.5cm}\paragraph{1.4.1 กำหนดสิทธิการใช้งานของแต่ละผู้ใช้}




\hspace{0cm}\textbf{\subsection{\textbf{ผู้ดูแลระบบโรงพยาบาล}}}

\hspace{1cm}\subsubsection{จัดการข้อมูลการเข้าสู่ระบบ}

\hspace{2.5cm}\paragraph{2.1.1 สามารถเข้าสู่ระบบ}

\hspace{2.5cm}\paragraph{2.1.2 สามารถแก้ไข Username และ Password}

\hspace{2.5cm}\paragraph{2.1.3 สามารถเพิ่ม ลบ ผู้ใช้งานในระบบ}

\hspace{2.5cm}\paragraph{2.1.4 สามารถออกจากระบบ}

\hspace{1cm}\subsubsection{จัดการข้อมูลโรงพยาบาล}

\hspace{2.5cm}\paragraph{2.2.1 สามารถเพิ่ม ลบ แก้ไขข้อมูลของโรงพยาบาล}

\hspace{1cm}\subsubsection{จัดการข้อมูลวอร์ด}

\hspace{2.5cm}\paragraph{2.3.1 สามารถเพิ่ม ลบ แก้ไขข้อมูลวอร์ดของโรงพยาบาล}

\hspace{1cm}\subsubsection{ตั้งค่าระดับพยาบาล}

\hspace{2.5cm}\paragraph{2.4.1 สามารถตั้งค่าระดับพยาบาลของโรงพยาบาล}

\hspace{1cm}\subsubsection{ดูสถิติการทำงานของพยาบาล}

\hspace{2.5cm}\paragraph{2.5.1 สามารถดูข้อมูลสถิติการทำงานโดยภาพรวมของพยาบาล}





\hspace{0cm}\textbf{\subsection{\textbf{ผู้อำนวยการโรงพยาบาล}}}

\hspace{1cm}\subsubsection{จัดการข้อมูลการเข้าสู่ระบบ}

\hspace{2.5cm}\paragraph{3.1.1 สามารถเข้าสู่ระบบ}

\hspace{2.5cm}\paragraph{3.1.2 สามารถแก้ไข Username และ Password}

\hspace{2.5cm}\paragraph{3.1.3 สามารถออกจากระบบ}

\hspace{1cm}\subsubsection{การอนุมัติ}

\hspace{2.5cm}\paragraph{3.2.1 สามารถอนุมัติตารางเวรของพยาบาล}

\hspace{2.5cm}\paragraph{3.2.2 สามารถอนุมัติการลาของพยาบาล}

\hspace{1cm}\subsubsection{ดูสถิติการทำงานของพยาบาล}

\hspace{2.5cm}\paragraph{3.3.1 สามารถดูข้อมูลสถิติการทำงานของพยาบาล}







\hspace{0cm}\textbf{\subsection{\textbf{หัวหน้าพยาบาล}}}

\hspace{1cm}\subsubsection{จัดการข้อมูลการเข้าสู่ระบบ}

\hspace{2.5cm}\paragraph{4.1.1 สามารถเข้าสู่ระบบ}

\hspace{2.5cm}\paragraph{4.1.2 สามารถแก้ไข Username และ Password}

\hspace{2.5cm}\paragraph{4.1.3 สามารถออกจากระบบ}

\clearpage

\hspace{1cm}\subsubsection{จัดตารางเวร}

\hspace{2.5cm}\paragraph{4.2.1 สามารถจัดตารางเวรของพยาบาล}

\hspace{2.5cm}\paragraph{4.2.2 สามารถแก้ไขตารางเวรของพยาบาล}

\hspace{2.5cm}\paragraph{4.2.3 สามารถนำตารางออกเป็นไฟล์ PDF}

\hspace{1cm}\subsubsection{การอนุมัติ}

\hspace{2.5cm}\paragraph{4.3.1 สามารถอนุมัติการแลกเวรของพยาบาล}

\hspace{2.5cm}\paragraph{4.3.2 สามารถอนุมัติการลาของพยาบาล}

\hspace{1cm}\subsubsection{ตารางเวร}

\hspace{2.5cm}\paragraph{4.4.1 สามารถดูตารางเวรของตัวเองได้}

\hspace{1cm}\subsubsection{แลกเวร}

\hspace{2.5cm}\paragraph{4.5.1 สามารถแลกเวรกับพยาบาลคนอื่น}

\hspace{2.5cm}\paragraph{4.5.2 สามารถดูประวัติการแลกเวร}

\hspace{2.5cm}\paragraph{4.5.3 สามารถดูความคืบหน้าของการแลกเวรได้}

\hspace{2.5cm}\paragraph{4.5.4 สามารถยกเลิกการแลกเวร}


\hspace{1cm}\subsubsection{การลา}

\hspace{2.5cm}\paragraph{4.6.1 สามารถขอลาได้}

\hspace{2.5cm}\paragraph{4.6.2 สามารถดูประวัติการลาของตัวเองได้}

\hspace{2.5cm}\paragraph{4.6.3 สามารถดูความคืบหน้าของการลาได้}

\hspace{2.5cm}\paragraph{4.6.4 สามารถยกเลิกการลาได้}



\hspace{0cm}\textbf{\subsection{\textbf{พยาบาล}}}

\hspace{1cm}\subsubsection{จัดการข้อมูลการเข้าสู่ระบบ}

\hspace{2.5cm}\paragraph{5.1.1 สามารถเข้าสู่ระบบ}

\hspace{2.5cm}\paragraph{5.1.2 สามารถแก้ไข Username และ Password}

\hspace{2.5cm}\paragraph{5.1.3 สามารถออกจากระบบ}

\hspace{1cm}\subsubsection{ขอเวร}

\hspace{2.5cm}\paragraph{5.1.1 สามารถขอเวรได้}

\hspace{2.5cm}\paragraph{5.1.2 สามารถดูการขอเวรของพยาบาลคนอื่น}

\hspace{1cm}\subsubsection{ตารางเวร}

\hspace{2.5cm}\paragraph{5.2.1 สามารถดูตารางเวรของตัวเองได้}

\hspace{1cm}\subsubsection{แลกเวร}

\hspace{2.5cm}\paragraph{5.3.1 สามารถแลกเวรกับพยาบาลคนอื่น}

\hspace{2.5cm}\paragraph{5.3.2 สามารถดูประวัติการแลกเวร}

\hspace{2.5cm}\paragraph{5.3.3 สามารถดูความคืบหน้าของการแลกเวรได้}

\hspace{2.5cm}\paragraph{5.3.4 สามารถยกเลิกการแลกเวร}

\hspace{1cm}\subsubsection{การลา}

\hspace{2.5cm}\paragraph{5.4.1 สามารถขอลาได้}

\hspace{2.5cm}\paragraph{5.4.2 สามารถดูประวัติการลาของตัวเองได้}

\hspace{2.5cm}\paragraph{5.4.3 สามารถดูความคืบหน้าของการลาได้}

\hspace{2.5cm}\paragraph{5.4.4 สามารถยกเลิกการลาได้}

\vspace{0cm}

\begin{table}[h]
    \captionsetup{justification=raggedright,singlelinecheck=false}
    \centering
    \fontsize{10}{17}\selectfont	
    \resizebox{\textwidth}{!}{%
        \begin{tabular}{l c c c c c}
            \hline
            \multicolumn{1}{c}{\multirow{2}{*}{ขอบเขตการทำงาน}} & \multicolumn{5}{c}{ระดับของผู้ใช้ในระบบ} \\ \cline{2-6} 
            \multicolumn{1}{c}{} & ผู้ดูแลระบบแอปพลิเคชัน & ผู้ดูแลระบบโรงพยาบาล & ผู้อํานวยการโรงพยาบาล & หัวหน้าพยาบาล & พยาบาล \\ \hline
            จัดการข้อมูลการเข้าสู่ระบบ & \checkmark & \checkmark & \checkmark & \checkmark & \checkmark\\ 
            จัดการโรงพยาบาล & \checkmark & \checkmark & & & \\ 
            จัดการสิทธิการใช้งาน &\checkmark & \checkmark& & & \\ 
            จัดการข้อมูลวอร์ด  & \checkmark & \checkmark & & & \\ 
            ดูข้อมูลสถิติการทำงาน &\checkmark &\checkmark &\checkmark & & \\ 
            ตั้งค่าระดับพยาบาล &\checkmark &\checkmark &  & & \\ 
            จัดตารางเวร & & & &\checkmark & \\ 
            การอนุมัติ & & &\checkmark &\checkmark & \\ 
            ตารางเวร & & & &\checkmark &\checkmark \\ 
            แลกเวร & & & &\checkmark & \checkmark\\ 
            การลา & & & &\checkmark &\checkmark\\ 
            ขอเวร & & & & &\checkmark \\ 
            \hline
        \end{tabular}%
    }
    \caption{แสดงขอบเขตการทำงานแบ่งตามประเภทของผู้ใช้ในระบบ}
\end{table}


\clearpage

\section{ขั้นตอนการดำเนินงาน}

\hspace{0cm}\subsection{การเสนอหัวข้อโครงงาน}

\hspace{0cm}\subsection{รวบรวมความต้องการของระบบ}

\hspace{0cm}\subsection{ศึกษาทฤษฎีที่เกี่ยวข้อง}

\hspace{0cm}\subsection{การวิเคราะห์และออกแบบระบบ}

\hspace{0cm}\subsection{การพัฒนาต้นแบบ Prototype}

\hspace{0cm}\subsection{การพัฒนาระบบ}

\hspace{0cm}\subsection{การทดสอบระบบและปรับปรุงแก้ไข}

\hspace{0cm}\subsection{สรุปผลการดำเนินงาน}

\hspace{0cm}\subsection{จัดทำรูปเล่มและนำเสนอโครงงาน}

\vspace{2cm}

\hspace{0cm}\section{แผนการดำเนินงาน}

\vspace{0.5cm}

\begin{table}[h]
    \centering
    \captionsetup{justification=raggedright,singlelinecheck=false}
    \resizebox{\textwidth}{!}{%
    \fontsize{10}{17}\selectfont	
    \begin{tabular}{l c c c c c c c}
    \hline
    \multicolumn{1}{c}{รายการ/กิจกรรม} & ม.ค. & ก.พ. & มี.ค. & เม.ย. & พ.ค. & มิ.ย. & ก.ค. \\ \hline
    การเสนอหัวข้อโครงงาน & &  & \checkmark & & & \\ 
    รวบรวมความต้องการของระบบ &\checkmark & & & & & & \\ 
    ศึกษาทฤษฎีที่เกี่ยวข้อง & &\checkmark & & & & & \\ 
    การวิเคราะห์และออกแบบระบบ & & \checkmark&\checkmark & & & & \\ 
    การพัฒนาต้นแบบ Prototype &\checkmark &\checkmark &\checkmark & & & & \\ 
    การพัฒนาระบบ & & &\checkmark &\checkmark &\checkmark & & \\ 
    การทดสอบระบบและปรับปรุงแก้ไข & & & & & &\checkmark & \\ 
    สรุปผลการดำเนินงาน & & & & & & &\checkmark \\ 
    จัดทำรูปเล่มและนำเสนอโครงงาน & & & & & & &\checkmark \\ 
    \hline
    \end{tabular}%
    }
    \caption{แผนการดำเนินงาน}
    \end{table}
\clearpage


\hspace{0cm}\section{อุปกรณ์ที่ใช้ดำเนินงาน}

\hspace{0cm}\subsection{ฮาร์ดแวร์ที่ใช้ในการพัฒนาระบบ}

\hspace{1cm}\subsubsection{คอมพิวเตอร์ส่วนบุคคล}

\hspace{2.5cm}\hangindent=4.9cm\hangafter=1\paragraph{1.1.1 CPU : Apple M1 chip 8-core CPU with 4 perform­ance cores and 4 efficiency cores}

\hspace{2.5cm}\hangindent=4.9cm\hangafter=1\paragraph{1.1.3 RAM : 16GB}

\hspace{2.5cm}\hangindent=4.9cm\hangafter=1\paragraph{1.1.4 Storage : 256GB SSD}

\hspace{2.5cm}\hangindent=4.9cm\hangafter=1\paragraph{1.1.4 OS : macOS Sonoma 14.2.1 }


\hspace{0cm}\subsection{ซอฟต์แวร์ที่ใช้ในการพัฒนาระบบ}

\hspace{1cm}\subsubsection{Figma}

\hspace{1cm}\subsubsection{Visual Studio Code}

\hspace{1cm}\subsubsection{MAMP}

\hspace{1cm}\subsubsection{Docker}

\hspace{1cm}\subsubsection{Git}

\section{ประโยชน์ที่คาดว่าจะได้รับ}



