\numberwithin{equation}{chapter}
\numberwithin{equation}{section}
\baselineskip=8mm
\chapter{บทนำ}

% \renewcommand{\thesubsection}{\thechapter.\arabic{subsection}}
\renewcommand{\thesubsection}{\arabic{subsection}.}
\renewcommand{\theequation}{\thesection.\arabic{equation}}
\renewcommand{\thesection}{}

\section{ที่มาและความสำคัญของปัญหา}

เนื่องจากสถานการณ์การแพร่ระบาดของโรคติดเชื้อไวรัสโคโรนา 2019 (Covid -19) จากเมืองอู่ฮั่น(Wuhan) มณฑลหูเป่ย(Hubei) ประเทศสาธารณรัฐประชาชนจีน ทำให้มีการแพร่ระบาดขยายเป็นวงกว้างอย่างรวดเร็วไปยังประเทศต่างๆทั่วโลกรวมถึงประเทศไทยด้วย ทำให้ส่งผลกระทบต่อสาธารณสุข เศรษฐกิจ สังคม \cite{wuhan:hupei}
พยาบาลเป็นหนึ่งในทีมบุคลากรทางการแพทย์ที่มีบทบาทเป็นด่านหน้าในการควบคุมและป้องกันการแพร่ระบาดของโรค เป็นผู้ปฏิบัติงานโดยตรงกับผู้ป่วยต้องเข้าไปสัมผัสใกล้ชิดกับผู้ป่วย พยาบาลต้องตระหนักและดูแลป้องกันตนเองไม่ให้ติดเชื้อต้องมาตรฐานอย่างเคร่งครัด \cite{nurse:covid}
ทำให้พยาบาลเกิดความเหนื่อย ความเครียดจากภาระงานที่เพิ่มมากขึ้น ในการปฏิบัติงานพยาบาลต้องดูแลผู้ป่วยใน Home Isolation ที่ต้องรับคำปรึกษาตลอดเวลาและขึ้นเวร Community Isolation โรงพยาบาลสนามโดยไม่ได้หยุดพัก \cite{nurse:Isolation}
เมื่อเปรียบเทียบจำนวนพยาบาลกับสัดส่วนประชากรโดยมากถึง 1 : 353 \cite{people:nurse} ความเหนื่อยล้าจากการทํางานของพยาบาลส่งผลกระทบต่อ ความปลอดภัยของผู้ป่วยที่ลดลง การดูแลเอาใจใส่ผู้ป่วยที่อาจไม่ดีเท่าที่ควร พยาบาลต้องทํางานต่อเนื่องกันยาวนานถึง 12 ชั่วโมงอาจทําให้เกิดอัตราความผิดพลาดจากการทํางานเพิ่มขึ้น \cite{ngeaun:nurse}
โดยปกติแนวทางในการจัดตารางเวรของพยาบาลจะยึด แนวทางการบริหารการจัดตารางเวรหรือผลัด การเบิกเงินค่าตอบแทนนอกเวลา และค่าเวรหรือผลัด ของพยาบาลวิชาชีพ พยาบาลเทคนิค ผู้ช่วยพยาบาล กระทรวงสาธารณสุข เป็นหลักแต่สามารถปรับแต่งการจัดจัดตารางเวรให้เหมาะสมได้ \cite{tarang:nurse}
โดยการจัดตารางเวรหัวหน้าพยาบาลจะเป็นผู้จัดทำ การจัดตารางเวรแบบเดิมจะใช้การจดบันทึกในการดาษและเนื่องจากปัจจุบันเทคโนโลยีได้เข้ามามีบทบาทอย่างมากจึงเกิดการประยุกต์ใช้เทคโนโลยีเพื่อจัดตารางเวรของพยาบาล อาทิเช่นแอปพลิเคชัน “Microsoft Excel” ที่มาช่วยในการจัดตารางให้ดูง่ายสามารถคำนวณวันเวลาได้ลดความยุ่งยากในการจัดเก็บเอกสาร \cite{excel:headnurse}

จากการจัดตารางเวรข้างต้นของหัวหน้าพยาบาลแสดงให้เห็นว่าการจัดตารางเวรแบบเดิมหรือการจัดตารางเวรโดยการใช้แอปพลิเคชัน “Microsoft Excel” ก็ยังคงเกิดปัญหาในหลายๆเรื่อง เช่น การจัดตารางเวรจัดไม่เท่ากัน โดยอาจเกิดการเองเอียง ความเหลื่อมล้ำ โดยหัวหน้าพยาบาลไม่ได้มีข้อมูลในการชี้แจงที่ชัดเจน การจัดตารางเวรอาจไม่ได้ตรงตามความต้องการของพยาบาล ทำให้พยาบาลอาจมีการแลกเวรจำนวนมากๆ ซึ่งส่งผลให้การจัดตารางเวรนั้นเปลี่ยนไม่ตรงตามวัตถุประสงค์ที่หัวหน้าพยาบาลต้องการตั้งแต่ตั้น ข้อมูลตารางเวรหากมีการเปลี่ยนแปลงโดยการแลกเวรจะต้องทำการอัปเดตซึ่งเป็นไปได้ยาก 

จากปัญหาข้างต้นเพื่อแก้ไขปัญหาดังกล่าว ผู้จัดทำมุ่งเน้นการพัฒนาระบบผ่านเว็บแอปพลิเคชันเพื่อให้พยาบาลสามารถเข้าถึงและจัดการตารางเวรได้อย่างง่ายดายผ่านอินเทอร์เน็ต ระบบนี้จะช่วยลดความยุ่งยากในการจัดตารางเวร ตอบสนองความต้องการของพยาบาลในการจัดตารางเวร อำนวยความสะดวกในการขอลาและการแลกเวร โดยสามารถอัปเดตข้อมูลตารางเวรได้ง่ายขึ้น โดยหัวหน้าพยาบาลจะมีข้อมูลเก่าในระบบที่สามารถนำมาวิเคราะห์เพื่อจัดตารางเวรหรือตอบคำถามให้พยาบาลได้ ซึ่งทำให้ลดการถกเถียงในการจัดตารางเวร และมีภาพรวมให้ผู้อำนวยการโรงพยาบาลได้ทราบก่อนที่จะอนุมัติตารางเวรได้จึงทำให้โรงพยาบาลมีระบบระเบียบ มีประสิทธิภาพมากยิ่งขึ้น

\section{{วัตถุประสงค์}}

\hspace{0cm}\subsection{เพื่อสร้างแอปพลิเคชันช่วยเหลือหัวหน้าพยาบาลและอำนวยความสะดวกให้พยาบาล}

\hspace{0cm}\subsection{เพื่อเป็นการศึกษาและพัฒนาเว็บแอปพลิเคชัน}

\section{แนวคิดและหลักการ}

\section{ขอบเขตการศึกษา}

\section{ขั้นตอนการดำเนินงาน}

\section{แผนการดำเนินงาน}

\section{อุปกรณ์ที่ใช้ดำเนินงาน}
