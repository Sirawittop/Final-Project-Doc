% การเขียนเอกสารอ้างอิง ให้ใช้รูปแบบ APA7 
% ลักษณะของการเขียนนี้ เป็นแบบ bibtex หากใช้รูปแบบอื่น
% ให้ปรับเป็นแบบ APA7 
%
% หนังสือทั่วไป ให้ใช้รูปแบบ โดยชื่อเรื่องต้องเป็นตัวเอียง ผู้แต่ง 1 - 20 คน ให้ใส่ชื่อทุกคน
%
% ผู้แต่ง. (ปีที่พิมพ์). \textit{ชื่อเรื่อง} (ครั้งที่พิมพ์ พิมพ์ครั้งที่ 2 เป็นต้นไป). สำนักพิมพ์.
%
% เช่น ภาสกร เนตรทิพย์วัลย์, พรพรรณ ภูสาหัส, และวิถี ธุระธรรม. (2565). \textit{การตรวจร่างกาย} (พิมพ์ครั้งที่ 2). พิมพ์ดีการพิมพ์.
% หรือ Kee, J. L., Marshall, S. M., \& Forrester, M. C. (2021). \textit{Clinical calculations} (9th ed.). Elsevier.
%
% ปล. ถ้าไม่ปรากฏปีที่พิมพ์ให้ใส่ (ม.ป.ป.) สำหรับภาษาไทย และ (n.d.) สำหรับภาษาอังกฤษ
%
% วารสารงานวิจัย ให้ใช้รูปแบบดังนี้
%
% ชื่อผู้เขียนบทความ. (ปีที่พิมพ์). ชื่อบทความ. \textit{ชื่อวารสาร}, \textit{เลขฉบับที่ หรือ Volume}(ฉบับที่ หรือ issue), หน้าแรก-หน้าสุดท้าย.
%
% เช่น บุศรา ชัยทัศน์. (2559). การดูแลผู้ป่วยโรงมะเร็งลำไส้ใหญ่. \textit{วารสารพยาบาลสภากาชาดไทย}, \textit{9}(1),19-33.
%หรือ Plows, J. F., Stanley, J. L., \& Vickers, M. H. (2018). The pathophysiology of gestational diabetes metllitus. \textit{International journal of molecular sciences}, \textit{19}(11), 3342.
% https:/doi.org/10.3390/ijms19113342 [arXiv:2210.07273 [astro-ph.HE]]
%
% ปล. ต้องใส่เลข doi ในรูปแบบลิงค์ https และอ้างอิง arXiv ใน [...] ด้วย ถ้ามี
%
% ดูเพิ่มเติมจาก https://tinyurl.com/4p6c5mf5

\begin{thebibliography}{99}
%\addcontentsline{toc}{chapter}{\numberline{}\textbf{บรรณานุกรม}}

\bibitem{Edmund:Dynamics}
Copeland, E.J., Sami, M., Tsujikawa, S. (2006).
Dynamics of dark energy.
\textit{Int. J. Mod. Phys. D} \textit{15}, 1753-1936.
\;[http://arxiv.org/abs/hep-th/0603057].

\bibitem{Planck 2015}
 Ade, P. A. R., Aghanim, N., Arnaud, M., Ashdown, M., Aumont, J., Baccigalupi, C. et al. [Planck Collaboration]. (2015).
\textit{Planck 2015 results. XIII. Cosmological parameters}.
\;[http://arxiv.org/abs/1502.01589].

\bibitem{wmap9} 
Hinshaw, G., Larson, D., Komatsu, E., Spergel, D. N., Bennett, C. L., Dunkley, J. et al. (WMAP Collaboration). (2012).
Nine-Year Wilkinson Microwave Anisotropy Probe (WMAP) Observations: Cosmological Parameter Results.
\textit{Astrophys. J. Suppl.}, \textit{208}(2), 19.
\;[arXiv:1212.5226 [astro-ph.CO]].

\bibitem{Power-law}
Gumjudpai, B. (2013).
Quintessential power-law cosmology: dark energy equation of state.
\textit{Mod. Phys. Lett. A.}, \textit{28}, 1350122.
\;[arXiv:1307.4552 [astro-ph.CO]].

\bibitem{intro:cosmology}
Liddle, A. (2015).
An Introduction to Mordern Cosmology (3rd).
London:Wiley.

\bibitem{st.uni}
Colin Barschel. (2007). Structure Formation in the Universe.

\bibitem{cdm model}
Dodelson, S., Gates, E., Turner, M. S. (1996).
Cold Dark Matter Models.
\textit{Science}, \textit{274}, 69-75.
https://doi.org/10.1126/science.274.5284.69
\;[arXiv:astro-ph/9603081].

\bibitem{bigbang}
Carlyle, T., \& Essayist, S. (1795-1881) The Big Bang Theory and Expansion of the Universe.
http://eldora.as.arizona.edu/~yshirley/Arizona/AST202/BigBang.pdf

\bibitem{thaicosmo}
บุรินทร์ กำจัดภัย. (2563).จักรวาลวิทยา: ปฐมบท (พิมพ์ครั้งที่ 3). ร้านพิษณุโลกดอทคอม.


\end{thebibliography}