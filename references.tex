% การเขียนเอกสารอ้างอิง ให้ใช้รูปแบบ APA7 
% ลักษณะของการเขียนนี้ เป็นแบบ bibtex หากใช้รูปแบบอื่น
% ให้ปรับเป็นแบบ APA7 
%
% หนังสือทั่วไป ให้ใช้รูปแบบ โดยชื่อเรื่องต้องเป็นตัวเอียง ผู้แต่ง 1 - 20 คน ให้ใส่ชื่อทุกคน
%
% ผู้แต่ง. (ปีที่พิมพ์). \textit{ชื่อเรื่อง} (ครั้งที่พิมพ์ พิมพ์ครั้งที่ 2 เป็นต้นไป). สำนักพิมพ์.
%
% เช่น ภาสกร เนตรทิพย์วัลย์, พรพรรณ ภูสาหัส, และวิถี ธุระธรรม. (2565). \textit{การตรวจร่างกาย} (พิมพ์ครั้งที่ 2). พิมพ์ดีการพิมพ์.
% หรือ Kee, J. L., Marshall, S. M., \& Forrester, M. C. (2021). \textit{Clinical calculations} (9th ed.). Elsevier.
%
% ปล. ถ้าไม่ปรากฏปีที่พิมพ์ให้ใส่ (ม.ป.ป.) สำหรับภาษาไทย และ (n.d.) สำหรับภาษาอังกฤษ
%
% วารสารงานวิจัย ให้ใช้รูปแบบดังนี้
%
% ชื่อผู้เขียนบทความ. (ปีที่พิมพ์). ชื่อบทความ. \textit{ชื่อวารสาร}, \textit{เลขฉบับที่ หรือ Volume}(ฉบับที่ หรือ issue), หน้าแรก-หน้าสุดท้าย.
%
% เช่น บุศรา ชัยทัศน์. (2559). การดูแลผู้ป่วยโรงมะเร็งลำไส้ใหญ่. \textit{วารสารพยาบาลสภากาชาดไทย}, \textit{9}(1),19-33.
%หรือ Plows, J. F., Stanley, J. L., \& Vickers, M. H. (2018). The pathophysiology of gestational diabetes metllitus. \textit{International journal of molecular sciences}, \textit{19}(11), 3342.
% https:/doi.org/10.3390/ijms19113342 [arXiv:2210.07273 [astro-ph.HE]]
%
% ปล. ต้องใส่เลข doi ในรูปแบบลิงค์ https และอ้างอิง arXiv ใน [...] ด้วย ถ้ามี
%
% ดูเพิ่มเติมจาก https://tinyurl.com/4p6c5mf5

\begin{thebibliography}{99}
    
\bibitem{wuhan:hupei}
\bibitem{nurse:covid}
\bibitem{nurse:Isolation}
\bibitem{people:nurse}
\bibitem{ngeaun:nurse}
\bibitem{tarang:nurse}
\bibitem{excel:headnurse}
\end{thebibliography}