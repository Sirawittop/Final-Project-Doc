
\begin{thebibliography}{99}

\bibitem{wuhan:hupei}
    กรมควบคุมโรค. (2563). \textbf{คู่มือเจ้าหน้าที่สาธารณะสุขในการโต้ตอบภาวะฉุกเฉิน กรณีการระบาดโรคติดเชื้อไวรัสโคโรนา 2019 ในประเทศไทย.} ม.ป.ท.: ม.ป.พ.
\bibitem{nurse:covid}
    นราจันทร์ ปัญญาวุทโส, ปรัชญานันท์ เที่ยงจรรยา และประภาพร ชูกำเหนิด. (2565). \\วารสารมหาวิทยาลัยคริสเตียน. \textbf{ประสบการณ์ของพยาบาลวิชาชีพในการมีส่วนร่วม\\ด้านความปลอดภัยในภาวะวิกฤตของการแพร่ระบาดโรคโควิด 19 โรงพยาบาลหาดใหญ่ ประเทศไทย,} 28, 59-72.
\bibitem{nurse:Isolation}
    คณะกรรมาธิการการสาธารณะสุข วุฒิสภา. (2565). \textbf{ภาระงานและประสิทธิภาพของวิชาชีพพยาบาล ภายใต้สถานะการณ์การระบาดของโรค COVID 19.} ม.ป.ท.: ม.ป.พ.
\bibitem{people:nurse}
    สำนักงานปลัดกระทรวงสาธารณะสุข กระทรวงสาธารณะสุข. (2564). \textbf{สัดส่วนเจ้าหน้าที่ทางการแพทย์ต่อประชากร.} ม.ป.ท.: ม.ป.พ.
\bibitem{ngeaun:nurse}
    เกศินี กิตติบาล, อารี ชีวเกษมสุข และชูชาติ พ่วงสมจิตร์. (2564). วารสารพยาบาลโรคหัวใจและทรวงอก. \textbf{การจัดการความเหนื่อยล้าจากการทํางานของพยาบาลวิชาชีพ \\โรงพยาบาลพระนครศรีอยุธยา,} 32, 121-136.
\bibitem{tarang:nurse}
    กองการพยาบาล กระทรวงสาธารณะสุข. (2566). \textbf{แนวทางการบริหารการจัดตารางเวรหรือผลัด \\การเบิกเงินค่าตอบแทนนอกเวลาและค่าเวรหรือผลัดของพยาบาลวิชาชีพ พยาบาลเทคนิค ผู้ช่วยพยาบาล กระทรวงสาธารณสุข.} ม.ป.ท.: ม.ป.พ.
\bibitem{excel:headnurse}
    ปริวัฒณ์ อารีชาติ และคณะ. (2565). Thai Journal of Operations Research: TJOR. \textbf{ตัวแบบการจัดตารางเวรของเภสัชกรเพื่อลดความเหลื่อมล้ําของภาระงาน,} 10, 103-112
\end{thebibliography}