
\baselineskip=8mm
\chapter{เอกสารและงานวิจัยที่เกี่ยวข้อง}

% \renewcommand{\thesubsection}{\thechapter.\arabic{subsection}}
\numberwithin{equation}{chapter}
\numberwithin{equation}{section}
\renewcommand{\thesubsection}{\arabic{subsection}.}
\renewcommand{\theequation}{\thesection.\arabic{equation}}
\renewcommand{\thesection}{}
\renewcommand{\thesubsubsection}{\thesubsection\arabic{subsubsection}.}



ในการศึกษา ภาคนิพนธ์เรื่อง “แอปพลิเคชันจัดตารางเวรพยาบาล” ผู้จัดทำได้ศึกษาหลักการ ทฤษฎี และงานวิจัยที่เกี่ยวข้อง เพื่อเป็นแนวทางในการศึกษาและพัฒนาภาคนิพนธ์ รายละเอียดหัวข้อดังต่อไปนี้

\hspace{0cm}\subsection{แนวทางการจัดตารางเวรและการเบิกเงินค่าตอบแทนนอกเวลาของพยาบาล}

\hspace{0cm}\subsection{ทฤษฎีการวิเคราะห์และออกแบบระบบ SDLC}

\hspace{0cm}\subsection{ภาษาและเครื่องมือที่ใช้ในการพัฒนา}

\hspace{0cm}\subsection{ทฤษฎีเกี่ยวกับความพึงพอใจ}

\hspace{0cm}\subsection{งานวิจัยที่เกี่ยวข้อง}

\clearpage

\renewcommand{\thesubsection}{}

\section{แนวทางการจัดตารางเวรและการเบิกเงินค่าตอบแทนนอกเวลาของพยาบาล}

อ้างถึงบันทึกของกระทรวงสาธารณะสุขที่ สธ 0202.3.7/ว 79 เรื่อง “ข้อบังคับกระทรวงสาธารณะสุขว่าด้วยการจ่ายเงินค่าตอบแทนเจ้าหน้าที่ที่ปฏิบัติงานให้กับหน่วยบริการในสังกัดกระทรวงสาธารณะสุข พ.ศ.2566 หลักเกณฑ์วิธีการและเงื่อนไขการจ่ายเงินค่าตอบแทน จำนวน 5 ฉบับ” ลงไว้เมื่อ 3 กุมภาพันธ์ 2566 ได้มีการปรับปรุงข้อบังคับฯ ดังกล่าว เพื่อให้มีความเหมาะสมกับภาวะเศรษฐกิจในปัจจุบัน และเพื่อให้เกิดความเข้าใจตรงกัน ในเรื่องการเบิกเงินค่าตอบแทนนออกเวลา (Over Time, OT) และค่าเวรหรือพลัดของพยาบาลวิชาชีพ พยาบาลเทคนิค ผู้ช่วยพยาบาล รวมทั้งให้เกิดความถูกต้องและเป็นธรรม ในเรื่องการบริหารจัดการชั่วโมงการทำงานของพยาบาล กระทรวงสาธารณสุข จึงมีหลักการและแนวทางปฎิบัติที่ผู้บริหารและผู้เกี่ยวข้องสามารถนำไปดำเนินการ ดังนี้

\hspace{0pt}\subsection{หลักการ}


\hspace{0.5cm}\hangindent=2.6cm\hangafter=1\subsubsection{เวรเช้า ผลัดบ่าย ผลัดดึก (เวรพลัดๆละ 8 ชั่วโมง) เป็นการปฏิบัติงานตามปกติของพยาบาลวิชาชีพ พยาบาลเทคนิค ผู้ช่วยพยาบาล เนื่องจากเป็นผู้ที่ปฏิบัติงานผลัดเปลี่ยนหมุนเวียนกันดูแลผู้ป่วยตลอด 24 ชั่วโมง}

\hspace{0.5cm}\hangindent=2.6cm\hangafter=1\subsubsection{การปฏิบัติงานของพยาบาลวิชาชีพ พยาบาลเทคนิค ผู้ช่วยพยาบาล แต่ละเดือนจะมีจำนวนเวรเท่ากับวันทำการในของเดือนนั้นๆ นับว่าเป็นการปฏิบัติงานโดยปกติของพยาบาล นอกเหนือจากจำนวนเวรดังกล่าวจึงเป็นการปฏิบัติงานนอกเวลา (Over Time, OT)}

\hspace{0.5cm}\hangindent=2.6cm\hangafter=1\subsubsection{ในแต่ละเวรหรือผลัด ควรกำหนดให้มีพยาบาลวิชาชีพที่มีศักยภาพการปฏิบัติงานต่างระดับอย่างน้อย 2 ระดับ (Skill mix) ขึ้นไป}

\hspace{0.5cm}\hangindent=2.6cm\hangafter=1\subsubsection{การเบิกงานค่าตอบแทนนอกเวลา (Over Time, OT) เบิกจากจำนวนเวรหรือผลัดที่เกินจากการจัดเวรหรือผลัดปกติ โดยสามารถเบิกได้ทั้งเวรเช้า หรือผลัดบ่าย หรือผลัดดึก}

\hspace{0.5cm}\hangindent=2.6cm\hangafter=1\subsubsection{การเบิกเงินค่าผลัดบ่าย หรือผลัดดึก มีจัดให้เป็นการเบิกเงินในการปฏิบัติงานปกติไม่ใช่การปฏิบัติงานนอกเวลา (Over Time, OT)}

\hspace{0.5cm}\hangindent=2.6cm\hangafter=1\subsubsection{เวรเช้า ผลัดบ่าย ผลัดดึก สามารถจัดเวรเสริมได้ตามภาระงานที่กำหนด และเวรเสริมนั้นจะเบิกเงินค่าตอบแทนนอกเวลา (Over Time, OT)}

\clearpage

\hspace{0pt}\subsection{นโยบายกลุ่มการพยาบาล}

\hspace{1cm}{เป็นนโยบายในการจัดเวรผลัดของพยาบาลวิชาชีพ พยาบาลเทคนิค ผู้ช่วยพยาบาล สำหรับทุกหน่วยงานให้ถือปฏิบัติ เพื่อความเป็นธรรมแก่พยาบาลและความปลอดภัยของผู้ป่วย ได้แก่}

\hspace{0.5cm}\hangindent=2.6cm\hangafter=1\subsubsection{ให้จัดทำตารางรูปแบบการจัดเวรแบบล็อคเวรผลัด เช่น ช/ช/ด/ด/บ/บ หรือ ช/ช/ด/ด/ด/บ หรือ ช/ช/ช/ด/บ/บ หรืออย่างอื่นเป็นต้น (เฉลี่ยเวรเข้า ผลัดบ่าย และผลัดดึก ให้พยาบาลแต่ละคนในหน่วยงานเท่าๆกัน)}

\hspace{0.5cm}\hangindent=2.6cm\hangafter=1\subsubsection{การแลกเวร/เปลี่ยนเวร จะสามารถแลกเปลี่ยนเวรได้ในพยาบาลที่มีศักยภาพการปฏิบัติงานในระดับเดียวกัน โดยได้รับอนุญาต และแก้ไขตารางเวรจากหัวหน้าหน่วยงานหรือผู้ที่ได้รับมอบหมายเท่านั้น}

\hspace{0.5cm}\hangindent=2.6cm\hangafter=1\subsubsection{การกำหนดเพิ่มและลดจำนวนพยาบาลที่ปฏิบัติงาน พิจารณาจากภาระงานซึ่งประกอบด้วยจำนวนผู้ป่วยและประเภทของผู้ป่วย}

\hspace{0.5cm}\hangindent=2.6cm\hangafter=1\subsubsection{หัวหน้าหน่วยงานตรวจสอบจำนวนชั่วโมงการทำงานต่อสัปดาห์ของพยาบาล โดยหลีกเลี่ยงการปฏิบัติงานากกว่า 60 ชั่วโมงต่อสัปดาห์ หรือมากกว่า 12 ชั่วโมงต่อเวรติดกันเกิน 3 วัน}

\hspace{0.5cm}\hangindent=2.6cm\hangafter=1\subsubsection{ต้องให้หัวหน้าหน่วยงานตรวจสอบและเซ็นรับรองตารางเวรแต่ละเดือน หลังจากพิจารณาว่าเป็นไปตามแนวทางการปฏิบัติข้อ 1-4}



\section{ทฤษฎีการวิเคราะห์และออกแบบระบบ SDLC}

\section{ภาษาและเครื่องมือที่ใช้ในการพัฒนา}

\section{ทฤษฎีเกี่ยวกับความพึงพอใจ}

\section{งานวิจัยที่เกี่ยวข้อง}

