% -\textbf{}%\documentclass[a4paper,12pt]{upthesis}
\documentclass[a4paper,12pt]{upthesis}
\usepackage{graphicx}
\usepackage{amsmath}
\usepackage{amssymb}
\usepackage{chngcntr}
\usepackage{booktabs}
\usepackage{colortbl}
%\usepackage{natbib}
% แพคเกจสำหรับสร้าง hyperlink ด้วยคำสั่ง \href{<url>}{<Text to show>}
%\usepackage[linktocpage=true]{hyperref}       % Create hyperlink with command \href{<url>}{<Text to show>}.

% แพคเกจสำหรับเรียกใช้การหมุนภาพ หรือตาราง
\usepackage{rotating}       % Package for rotating either figure or table with command
                                      %\begin{sidewaysfigure} or \begin{sidewaystable} and also \end{} as well.
\usepackage{enumitem}		% For customize the symbols or enumeration                                      

% เป็นการกำหนดค่าเลขหน้าภาษาไทยในช่วงแรก
\usepackage{polyglossia}
\setdefaultlanguage{thai}
\setotherlanguages{thai}
% เป็นการกำหนดแพคเกจสำหรับจัดการเรื่องฟอนต์ รวมถึงการใช้ภาษาไทย
\usepackage{fontspec}
\usepackage{xunicode}
\usepackage{xltxtra}

\usepackage{indentfirst}
\usepackage{fancybox}
\usepackage{pst-grad}
\usepackage{pst-osci}
\usepackage{pst-coil}
\usepackage{tabularx}

% Set caption of Table & Figure align on the left most
% สำหรับกำหนดคำอธิบายของตารางและรูปให้อยู่ด้านซ้ายสุด
\usepackage[labelfont=bf, textfont=bf]{caption}
% labelfont = bf เป็นการกำหนดให้คำนำของตารางและรูปเป็นตัวหนา เช่น ตารางที่ 2 หรือ รูปที่ 3 จะเป็นตัวหนา เป็นต้น
% textfont = bf เป็นการกำหนดตัวอักษรคำอธิบายเป็นตัวหนา
% textfont = sl เป็นการกำหนดตัวอักษรคำอธิบายเป็นตัวเอียง
% singlelinecheck=off เป็นการกำหนดคำอธิบายที่มีบรรทัดเดียวให้เริ่มที่ด้านซ้ายสุด

% Set caption of Table & Figure align at start of Table or Figure LHS.
% กำหนดคำอธิบายตารางและภาพให้เริ่มต้นที่ด้านซ้ายมือสุดของตาราง หรือรูปภาพ
\usepackage{floatrow}
\floatsetup[table]{style=plaintop, footnoterule=none,footskip=.35\skip\footins}  
% Set caption of Table on top of table
% กำหนดคำอธิบายตารางให้อยู่ด้านบนของตาราง
% Use command \ttabbox {\caption {<caption text here>}} to set text equal a width of table.
% ใช้คำสั่ง \ttabbox {\caption {<caption text here>}} เพื่อกำหนดให้คำอธิบายมีขนาดความกว้างเท่ากับความกว้างของตาราง

% กำหนดการตัดคำ ขนาด และฟอนต์ที่จะใช้สำหรับภาษาไทย
\XeTeXlinebreaklocale "th_TH"
\XeTeXlinebreakskip = 0pt plus 1pt
\defaultfontfeatures{Scale=1.23}
\setmainfont[Mapping=tex-text]{TH Sarabun New}

%\setdefaultlanguage{thai}
%\newfontfamily{\thaifont}[Script=thai]{TH Niramit AS}

\usepackage{pstricks,pst-plot,pst-node}
\usepackage{pst-3dplot}
\usepackage{pstricks-add}

\usepackage{makeidx}         % allows index generation
\usepackage{graphicx}        % standard LaTeX graphics tool
                                             % when including figure files
\usepackage{multicol}         % used for the two-column index
\usepackage[bottom]{footmisc}% places footnotes at page bottom
\usepackage{tensor}             %for write tensor indices
\usepackage{color}                 % for assign a color to text with command
                                        %\textcolor{color}{text}
\usepackage{enumitem}		% For customize the symbols or enumeration                           

%Package for setup page (paper size)
% สำหรับการกำหนดขนาดของกระดาษ   *** แก้ไขได้ หากใช้กระดาษ และจัดขอบกระดาษต่างจากนี้ ***
\usepackage{geometry}
 \geometry{
 a4paper,		% Paper size ขนาดกระดาษ
 total={147.5mm,234.5mm},	% Total area of text on the page	ขนาดกว้างยาวของพื้นที่สำหรับข้อความในกระดาษ
 left=37.5mm,			% Left margin	1 inch = 25.4 mm ระยะขอบซ้าย
 %right=25mm,				% Right margin ระยะขอบขวา (ไม่กำหนดก็ได้ เพราะจะจัดอัตโนมัติ)
 top=37.5mm,			% Top margin	1.5 inch = 38.1 mm ระยะขอบบน
% bottom=25mm,		% Bottom margin ระยะขอบล่าง (ไม่กำหนดก็ได้ เพราะจะจัดอัตโนมัติ)
 }


% กำหนดรูปแบบสิ่งแวดล้อมของตัวอย่าง  *** ไม่ควรแก้ไข ***
\newcounter{Exampctr}\newsavebox{\Exampname}
\newenvironment{example}[1]
  {\stepcounter{Exampctr}%
   \sbox{\Exampname}{\textit{#1}}
   \begin{description}\item[ตัวอย่างที่ \arabic{Exampctr}][\usebox{\Exampname}]~}
   {\end{description}}

% กำหนดรูปแบบสิ่งแวดล้อมของนิยาม
\newcounter{Defctr}\newsavebox{\Defname}
\newenvironment{definition}[1]
  {\stepcounter{Defctr}%
   \sbox{\Defname}{\textit{#1}}
   \begin{description}\item[นิยาม \arabic{Defctr}][\usebox{\Defname}]~}
   {\end{description}}

% กำหนดรูปแบบสิ่งแวดล้อมของแบบฝึกหัด
\newcounter{Probctr}\newsavebox{\Probname}
\newenvironment{problem}[1]
  {\stepcounter{Probctr}
   \sbox{\Probname}{\textit{#1}}
   \begin{description}\item[แบบฝึกหัดที่ \arabic{Probctr}][\usebox{\Probname}]~}
   {\end{description}}

% กำหนดรูปแบบสิ่งแวดล้อมของทฤษฎีบท
\newcounter{Theoctr}\newsavebox{\Theoname}
\newenvironment{theorem}[1]
  {\stepcounter{Theoctr}
   \sbox{\Theoname}{\textit{#1}}
   \begin{description}\item[ทฤษฎีบทที่ \arabic{Probctr}][\usebox{\Theoname}]~}
   {\end{description}}

% การนิยามคำสั่งที่ใช้บ่อย และยาว ให้สั้นลง
\def\l{\left}
\def\r{\right}
%\def\be{\begin{equation}}
%\def\ee{\end{equation}}
%\def\bea{\begin{eqnarray}}
%\def\eea{\end{eqnarray}}
\def\f{\frac}
\def\no{\nonumber}
\def\i{\hat{i}}
\def\j{\hat{j}}
\def\k{\hat{k}}
\def\ro{\hat{\rho}}
\def\vp{\hat{\varphi}}
\def\x{\hat{x}}
\def\y{\hat{y}}
\def\z{\hat{z}}
\def\e#1{\hat{e}_{#1}}
\def\rh{\hat{r}}
\def\tt{\hat{\theta}}
\def\ph{\hat{\phi}}
\def\cross{\times}
\def\pd{\partial}
\def\pdx{\frac{\partial}{\partial x}}
\def\pdy{\frac{\partial}{\partial y}}
\def\pdz{\frac{\partial}{\partial z}}
\def\pdt{\frac{\partial}{\partial t}}
\def\ddx{\frac{d}{dx}}
\def\ddy{\frac{d}{dy}}
\def\ddz{\frac{d}{dz}}
\def\ddt{\frac{d}{dt}}
\def\pt#1#2{\frac{\partial{#1}}{\partial{#2}}}
\def\dd#1#2{\frac{d{#1}}{d{#2}}}
\def\det#1{\rm det({#1})}
\def\tr#1{\rm Tr({#1})}
\def\ul#1{\rm \underline{#1}}
\def\dul#1{\rm \underline{\underline{#1}}}
\def\mi#1#2{\rm M_{#1}(\underline{#2})}
\def\co#1#2{\rm C_{#1}(\underline{#2})}
\def\red#1{\textcolor{red}{#1}}
\def\blue#1{\textcolor{blue}{#1}}
\def\w9{{\rm WMAP9+BAO+}$H_0$}
\def\p15{$TT,TE,EE${\rm +lowP+Lensing+ext}}


% More references in gloss-thai of polyglossia package
% http://texdoc.net/texmf-dist/tex/latex/polyglossia/gloss-thai.ldf
% \thaiAlpha works well, the sequence is ก ข "ฃ" ค "ฅ" ฆ ง จ ...
% Normally, we need ก ข ค ง จ which is defined in \thaialph
% I'm not sure why it doesn't work. So, I just re-define it.
\def\thaialph#1{\expandafter\thalph\csname c@#1\endcsname}
\def\thalph#1{%
    \ifcase#1\or ก\or ข\or ค\or ง\or จ\or ฉ\or ช\or ซ\or
    ฌ\or ญ\or ฎ\or ฏ\or ฐ\or ฑ\or ฒ\or ณ\or ด\or ต\or ถ\or ท\or ธ\or น\or
    บ\or ป\or ผ\or ฝ\or พ\or ฟ\or ภ\or ม\or ย\or ร\or ฤ\or ล\or ฦ\or ว\or
    ศ\or ษ\or ส\or ห\or ฬ\or อ\else ฮ\else\xpg@ill@value{#1}{thalph}\fi}

\usepackage{subcaption}
%\renewcommand{\figurename}{รูปที่}
% กำหนดให้ subfigurename เป็น (ก), (ข), (ค), ...
\renewcommand{\thesubfigure}{\thaialph{subfigure}}	

%\renewcommand{\chaptername}{บทที่}
%\renewcommand{\figurename}{รูปที่}
%\renewcommand{\contentsname}{สารบัญ}
%\renewcommand{\indexname}{ดัชนี}

%%%%%%%%%%%%%%%
% Add prefix in front of Chapter number in Content
\usepackage{afterpage}
\DeclareRobustCommand*{\contheading}{%
  \afterpage{{\normalfont\large\bfseries\centering
 {\large\bf สารบัญ (ต่อ)}\\
 \hspace{10mm}{\normalsize\bf บทที่}\hfill {\normalsize\bf หน้า}\par\vspace{0.7cm}}}}
%%%%%%%%%%%%%%%
\DeclareRobustCommand*{\figcontheading}{%
  \afterpage{{\normalfont\large\bfseries\centering
 {\large\bf สารบัญภาพ (ต่อ)}\\ \vspace{0.5cm}
 \hspace{10mm}{\normalsize\bf ภาพ}\hfill {\normalsize\bf หน้า}\par\vspace{0.7cm}}}}
 %%%%%%%%%%%%%%%
\DeclareRobustCommand*{\tabcontheading}{%
  \afterpage{{\normalfont\large\bfseries\centering
 {\large\bf สารบัญตาราง (ต่อ)}\\ \vspace{0.5cm}
 \hspace{10mm}{\normalsize\bf ตาราง}\hfill {\normalsize\bf หน้า}\par\vspace{0.7cm}}}}
 %%%%%%%%%%%%%%%%%%%%%%%%%%%%%%

%%%%%%%%%%%%%%%%%%% Make List of Figure (Continued) page %%%%%%%%%%%%%%%%%%
\makeatletter



%%%%%%%%%%%%%%%%%%%%%%%%%%%%%%%%%%%%%%%

% สร้างดัชนีคำ (ถ้ามี) ก็ให้เอา % หน้า \makeindex ออก
%\makeindex             % used for the subject index
                       % please use the style svind.ist with
                       % your makeindex program

% สร้างเส้นหนาให้กับเส้นแนวนอน และเส้นแนวตั้งของตาราง
% ด้วยคำสั่ง \thline{1pt} หรือ \tvline{1pt}   %***ไม่ควรแก้ไข ***
% Make a Thick hline and vline in a Table
% by using \thline{1pt} or \tvline{1pt}
\usepackage{array,tabularx}
\makeatletter
\def\thline#1{%
    \noalign {\ifnum 0=`}\fi \hrule height #1
    \futurelet \reserved@a \@xhline
}
\def\tvline#1{@{\hskip\tabcolsep\vrule width #1\hskip\tabcolsep}}
\makeatother
% End of make thick hline and vline

% package{multirow} ใช้รวมแถวในตาราง
%package{multirow} is use to combine the rows in a table.
\usepackage{multirow}
% package{array} ใช้ขยายความสูงของแต่ละแถวในตาราง
%package{array} is use to expand the height of each row in a table.
\usepackage{array}
%This command is use to setting the stretch of the height of each row in a table.
% คำสั่งสำหรับตั้งค่าความสูงของแต่ละแถวในตาราง
%\renewcommand{\arraystretch}{2}
%\usepackage{fancyhdr}   % Customized the heading with command \lhead{}, \rhead{}, \chead{}, \lfoot{}, \rfoot{}, \cfoot{}
%\DeclareRobustCommand*{\contheading}{%
%  \afterpage{{\normalfont\large\bfseries\centering
%  สารบัญ (ต่อ)\par\bigskip}}}
% spacing after headings
%the titlesec package- it provides the command \titlespacing which has the format
%\titlespacing{command}{left spacing}{before spacing}{after spacing}[right]
%From the titlesec package
% spacing: how to read {12pt plus 4pt minus 2pt}
%           12pt is what we would like the spacing to be
%           plus 4pt means that TeX can stretch it by at most 4pt
%           minus 2pt means that TeX can shrink it by at most 2pt
%       This is one example of the concept of, 'glue', in TeX
%\titlespacing\section{0pt}{12pt plus 4pt minus 2pt}{0pt plus 2pt minus 2pt}

%%%%%%%%%%%%%%%%%%%%%%%%%%%%%%%%%%%%%%%%%%%%%%%%%%%%%%%%%%%
%%    ผู้เขียนเล่มการศึกษาอิสระ สามารถทำการแก้ไขได้ตั้งแต่ส่วนนี้เป็นต้นไป
%%%%%%%%%%%%%%%%%%%%%%%%%%%%%%%%%%%%%%%%%%%%%%%%%%%%%%%%%%%

% กำหนดหัวข้อภาษาไทย

% Just only Thai Title!!! This will be used at the title of Abstract
\title{  						% Title in Thai ชื่อเรื่องภาษาไทย
แอปพลิเคชันจัดตารางเวรพยาบาล
}
\titleen{						% Title in English ชื่อเรื่องภาษาอังกฤษ
Application for Nursing shift scheduling 
}

% คำนำหน้าผู้เขียน
\nameprefix{นาย}
\nameprefixen{Mr.}

% ชื่อ และนามสกุลของผู้เขียน
% \author{Name}{Surname}
\author{สิรวิชญ์}{คำชุ่ม}
\authoren{Sirawit}{Kamchoom}


% ชื่ออาจารย์ที่ปรึกษา
\advisor{อาจารย์ธรรมรัตน์ ธรรมา}                    % advisor
\advisoren{Dr.Phongsaphat Rangdee}

% ชื่ออาจารย์ที่ปรึกษาร่วม
\coadvisor{อาจารย์วรกฤต แสนโภชน์}
\coadvisoren{Dr.Chonticha Kritpetch}

%ชื่อปริญญา
\degree{วิทยาศาสตรบัณฑิต}                                   % degree
\degreeen{Bachelor of Science}
%ชื่อย่อปริญญา
\degreeAbb{วท.บ.}                                               % degree (abbreviation)
\degreeAbben{B.Sc.}

%หลักสูตร หรือสาขาวิชา
\major{วิทยาการคอมพิวเตอร์}                                      % major
\majoren{Computer Science}

% ภาควิชา หรือสาขาวิชา
\dept{ภาควิชาฟิสิกส์}                      % department

% เดือน และปี พ.ศ. ที่สำเร็จการศึกษา
\date{ตุลาคม 2565}                                               % date

% ปีการศึกษา
\academicyear{2565}                                             % academic year

% คำสำคัญ ภาษาไทย  และภาษาอังกฤษ
\keywordsth{                                                      % keywords in Thai คำสำคัญ ภาษาไทย ควรมีแค่ 3-4 คำ
            จัดตารางเวรพยาบาล, จัดตาราง, เวรพยาบาล , เว็บแอปพลิเคชัน, พยาบาล
         }
\keywords{						% Keywords in English คำสำคัญ ภาษาอังกฤษ ไม่ควรยาวมาก และควรมีแค่ 3-4 คำ
			Chaplygin Gas, Dark Energy, Accelerating Universe, Phantom Power-Law
		}

%%%%%%%%%%%%%%%%%%%%%%%%%%%%%%%%%%%%%%%%%%%%%%%%%%%%%%%%%%%%%%%%%%%%%%%%
%%-----------------------------------------------
%%               some Biography                %%
%%            ข้อมูลประวัติส่วนตัว					%%
%%-----------------------------------------------
% your birth date วันเดือนปีเกิด
\dateofbirth{20 มิถุนายน 2546}
% place of your birth  สถานที่เกิด
\placeofbirth{จังหวัดแม่ฮ่องสอน, ประเทศไทย}
% present address (province and country)	จังหวัดปัจจุบัน และประเทศ
\provinceTH{จังหวัดเชียงราย, ประเทศไทย}                  
% your current address	ที่อยู่ตามทะเบียนบ้าน หรือที่ติดต่อได้
\address{เลขที่ 182 หมู่ 6 ตำบลผางาม อำเภอเวียงชัย จังหวัดเชียงราย}
\zipcode{57210}		% รหัสไปรษณีย์
% Phone number  เบอร์โทรปัจจุบัน
\myphone{099-6633516}
% Email อีเมลที่ใช้อยู่
\myemail{sirawit.code@gmail.com}

% your working place สถานที่ทำงาน (ถ้ามี)
%\workplace{}
%\position{Lecturer at University of Phayao}
%\workplace{Department of Physics, Faculty of Science, University of Phayao, Phayao 56000, Thailand}

% Work Experiences ประสบการณ์การทำงาน (ถ้ามี)
%\WorkExpr{      
%\workexpr{2005 - 2009}{Lecturer at Department of Physics, Faculty of Science, University of Phayao, Phayao, Thailand}
%\workexpr{2000 - 2001}{Lecturer at Department of Physics, Faculty of Science, Naresuan University, Phitsanulok, Thailand}
%}

% Education Background การศึกษาที่ผ่านมา
\EduBG{         
\authorEduBG{มัธยมศึกษาตอนต้น}{โรงเรียนสามัคคีวิทยาคม อำเภอเมือง จังหวัดเชียงราย}{2560}
\authorEduBG{มัธยมศึกษาตอนปลาย}{โรงเรียนสามัคคีวิทยาคม อำเภอเมือง จังหวัดเชียงราย}{2563}
}

%Publication List  รายการผลงานตีพิมพ์ (ถ้ามี)
%\Publication{   
%\publication{1}{\textbf{Rachan Rangdee} (IF) and Burin Gumjudpai (IF and ThEP), ``Tachyonic (phantom) power-law cosmology'', Astrophysics and Space Science \textbf{349} (2014) 975-984.
%[arXiv:1210.5550 [astro-ph.CO]]}
%\publication{2}{Burin Gumjudpai (IF and ICTP) and \textbf{Phongsaphat Rangdee} (IF), ``Non-minimal derivative coupling gravity in cosmology'', General Relativity and Gravitation \textbf{47} (2015) 140.
%[arXiv:1511.00491 [gr-qc]]}
%}

% Oral and Poster Presentation List  รายการการนำเสนอทั้งแบบ Oral & Poster
%\Presentation{          
%\presentation{1}{\textbf{Rachan Rangdee} and Burin Gumjudpai (IF), ORAL ``Non-minimal Derivative Coupling with Power-Law'', RGJ-Ph.D. Congress XVI ``ASEAN: Emerging Research Oppotunities'', Pattaya, Chonburi, Thailand (2015).}
%\presentation{2}{\textbf{Rachan Rangdee} (IF), ORAL ``Aspects of non-minimal derivative coupling'', the $7^{\rm th}$ Aegean Summer School Beyond Einstein's Theory of Gravity, Parikia, Island of Paros, Greece, (2013).}
%\presentation{3}{\textbf{Rachan Rangdee} (IF), ORAL ``Tachyonic Phantom Power-Law Cosmology'', the Workshop on Heavy Ion and High Energy Physics 2012 (HIHEP2012), Naresuan University, Phitsanulok, Thailand (2012).}
%\presentation{4}{\textbf{Rachan Rangdee} (IF), ORAL ``Tachyonic Power-Law Cosmology'', the Siam Physics Congress 2012 (SPC2012): Past, Present and Future of Physics, Phranakhon Si Ayutthaya, Thailand (2012).}
%\presentation{5}{\textbf{Rachan Rangdee} (TPTP), POSTER ``TACHYONIC DARK ENERGY MODEL: DYNAMICS AND EVOLUTION'', the $11^{\rm th}$ Asian-Pacific Regional IAU Meeting (APRIM2011), Chiang Mai, Thailand (2011).}
%}

%%%%%%%%%%%%%%%%%%%%%%%%%%%%%%%%%%%%%%%%%%%%%%%%%%%%%%%%%%%%%%%%%%%%%%
%%%  เริ่มต้นเอกสาร
%%%%%%%%%%%%%%%%%%%%%%%%%%%%%%%%%%%%%%%%%%%%%%%%%%%%%%%%%%%%%%%%%%%%%%

\begin{document}

\pagenumbering{thaialph}
%\maketitle
%\newpage
%%%%%%%%%%%%%%%%%%%%%%%%%%%
%%%
%%% MAKE A COVER PAGE
%%% สร้างหน้าปก (หน้าแรก)
%%%%%%%%%%%%%%%%%%%%%%%%%%%%%

\thispagestyle{empty}
\begin{center} % กำหนดชื่อเรื่อง ภาษาไทย และภาษาอังกฤษ
\Large{\textbf{แอปพลิเคชันจัดตารางเวรพยาบาล\\
(Application for Nursing shift scheduling)\\}}
\vspace{7cm}
\textbf{นายสิรวิชญ์ คำชุ่ม\\} % ชื่อนามสกุลผู้เขียน
\vspace{7cm}
\textbf{ภาคนิพนธ์เสนอมหาวิทยาลัยพะเยา เพื่อเป็นส่วนหนึ่งของการศึกษา\\
หลักสูตรปริญญาตรี วิทยาศาสตร์บัณฑิต\\
สาขาวิชาวิทยาการคอมพิวเตอร์\\
ลิขสิทธิ์เป็นของมหาวิทยาลัยพะเยา}
\end{center}

%%%%%%%%%%%%%%%%%%%%%%%%%%%%%%%%%%%%%%%%%%%%%%%%%%%%%%%%%%%%%%%%%%%%%%%%

%%%%%%%%%%%%%%%%%%%%%%%%%%%
%%%
%%% MAKE AN APPROVE PAGE สร้างหน้าอนุมัติ
%%%
%%%%%%%%%%%%%%%%%%%%%%%%%%%%%

% \newpage
% \setcounter{page}{1}
% \thispagestyle{empty}

% % ใส่ชื่อเรื่องการศึกษาอิสระ
% อาจารย์ที่ปรึกษาและประธานหลักสูตรวิทยาศาสตร์บัณฑิต สาขาวิชาวิทยาการคอมพิวเตอร์
% คณะเทคโนโลยีสารสนเทศและการสื่อสาร มหาวิทยาลัยพะเยาได้พิจารณาภาคนิพนธ์ เรื่อง “แอปพลิเคชันจัดตารางเวรพยาบาล” เห็นสมควรรับ เป็นส่วนหนึ่งของการศึกษารายวิชา 225492 โครงงานวิทยาการคอมพิวเตอร์ ภาคการศึกษาต้น ปีการศึกษา 2567 มหาวิทยาลัยพะเยา
% %Has been approved by the Graduate School as partial fulfillment of the requirements
% %for the Doctor of Philosophy Degree in Theoretical Physics\\ of Naresuan University.\\
% \vskip0.6cm
% \normalsize
% \begin{center}
% %\textbf{คณะกรรมการสอบ}\vskip1cm
% %\textbf{Oral Defense Committee}\vskip1cm
% \vskip2.0cm

% ...........................................................................\\
% (อาจารย์ธรรมรัตน์ ธรรมา)\\		% ใส่ชื่อประธานกรรมการสอบ
% อาจารย์ที่ปรึกษา \\ 
% \vskip2.0cm

% ...........................................................................\\
% (ดร.กนกวรรธน์ เซี่ยงเจ็น)\\		% ใส่ชื่อประธานกรรมการสอบ
% อาจารย์ที่ปรึกษา \\ 
% \vskip2.0cm

% ...........................................................................\\
% (อาจารย์วรกฤต แสนโภชน์)\\			% ใส่ชื่อกรรมการ คนที่เป็นอาจารย์ที่ปรึกษาหลัก
% กรรมการ \\ 	
% \vskip2.0cm

% ...........................................................................\\
% (อาจารย์วรกฤต แสนโภชน์)\\			% ใส่ชื่อกรรมการ คนที่เป็นอาจารย์ที่ปรึกษาร่วม 
% กรรมการ \\ 
% \vskip2.0cm


% %\textbf{ตรวจสอบแล้ว}\vskip1cm

% ...........................................................................\\
% (อาจารย์ธนวัฒน์ แซ่เอียบ)\\
% ประธานหลักสูตรวิทยาศาสตร์บัณฑิต สาขาวิทยาการคอมพิวเตอร์\\
% คณะเทคโนโลยีสารสนเทศและการสื่อสาร มหาวิทยาลัยพะเยา \\
% %\afterpage{\blankpage}
% %\copyrightpage
% \end{center}
% \clearpage

%%%%%%%%%%%%%%%%%%%%%%%%%%%%%%%%%%%%%%%%%%%%%%%%%%%%%%%%%%%%%%%%%%%%%%%%

%%%%%%%%%%%%%%%%%%%%%%%%%%%
%%%
%%% MAKE AN ACKNOWLEDGEDMENT PAGE
%%% สร้างกิตติกรรมประกาศ จะขอบคุณใคร อะไร ก็ใส่ตรงนี้
%%%%%%%%%%%%%%%%%%%%%%%%%%%%%
% \newpage
% %\addtocontents{toc}{\protect\contentsline{chapter}{กิตติกรรมประกาศ}{\thepage}\protect}
% \begin{acknowledgement}     % acknowledgement environment, this is optional
% \baselineskip=8mm
% \setcounter{page}{2}

% ในการศึกษาวิจัยครั้งนี้ ข้าพเจ้าขอขอบพระคุณ ...

% \vspace{0.5cm}
% % or \input{acknowledgement.tex} % you need a separate acknowledgement.tex file to include it.
% \begin{flushright}
% \makeauthor
% \end{flushright}

% \end{acknowledgement}


% \renewcommand{\arraystretch}{1.1}

%%%%%%%%%%%%%%%%%%%%%%%%%%%%%%%%%%%%%%%%%%%%%%%%%%%%%%%%%%%%%%%%%%%%%%%%

%%%%%%%%%%%%%%%%%%%%%%%%%%%
%%%
%%% MAKE AN ABSTRACT PAGE
%%% สร้างบทคัดย่อ ภาษาไทย และภาษาอังกฤษ
%%%%%%%%%%%%%%%%%%%%%%%%%%%%%
% \newpage
% %\addtocontents{toc}{\protect\contentsline{chapter}{บทคัดย่อ}{\thepage}\protect}
% \begin{abstractth}	% เขียนบทคัดย่อภาษาไทย

% บทคัดย่อภาษาไทย เขียนตรงนี้


% \end{abstractth}

% \newpage

% %\maketitle
% \renewcommand{\arraystretch}{1}

%%%%%%%%%%%%%%%%%%%%%%%%%%%%%%%%%%%%%%%%%%%%%%%%%%%%%%%%%%%%%%%%%%%%%%%%

%\newpage
% ถ้ามีคำนำ ก็สามารถเรียกใช้ตรงนี้ได้เลย
%\include{preface}   

%%%%%%%%%%%%%%%%%%%%%%%%%%%%%%%%%%%%%%%%%%%%%%%%%%%%%%%%%%%%%%%%%%%%%%%%
\newpage
% สร้างสารบัญ
\tableofcontents
% เพิ่มสารบัญเรื่อง ในสารบัญ
\clearpage

%%%%%%%%%%%%%%%%%%%%%%%%%%%%%%%%%%%%%%%%%%%%%%%%%%%%%%%%%%%%%%%%%%%%%%%%

% List of Tables
% สร้างสารบัญตาราง
\listoftables
% เพิ่มสารบัญตารางในสารบัญ
\clearpage

%%%%%%%%%%%%%%%%%%%%%%%%%%%%%%%%%%%%%%%%%%%%%%%%%%%%%%%%%%%%%%%%%%%%%%%%

%List of Figures
% สร้างสารบัญภาพ
\listoffigures
% เพิ่มสารบัญภาพในสารบัญ
\clearpage

%%%%%%%%%%%%%%%%%%%%%%%%%%%%%%%%%%%%%%%%%%%%%%%%%%%%%%%%%%%%%%%%%%%%%%%%

% กำหนดเลขหน้าเป็นเลขอารบิก
\pagenumbering{arabic}
\pagestyle{headings}
\setcounter{page}{1}

\newpage

%%%%%%%%%%%%%%%%%%%%%%%%%%%%%
%%%
%%% ADD CHAPTER TO THE MAIN FILE เพิ่มเนื้อหาของแต่ละบท
%%%
%%%%%%%%%%%%%%%%%%%%%%%%%%%%%

\numberwithin{equation}{chapter}
\numberwithin{equation}{section}
\baselineskip=8mm
\chapter{บทนำ}

% \renewcommand{\thesubsection}{\thechapter.\arabic{subsection}}
\renewcommand{\thesubsection}{\arabic{subsection}.}
\renewcommand{\theequation}{\thesection.\arabic{equation}}
\renewcommand{\thesection}{}

\section{ที่มาและความสำคัญของปัญหา}

เนื่องจากสถานการณ์การแพร่ระบาดของโรคติดเชื้อไวรัสโคโรนา 2019 (Covid -19) จากเมืองอู่ฮั่น(Wuhan) มณฑลหูเป่ย(Hubei) ประเทศสาธารณรัฐประชาชนจีน ทำให้มีการแพร่ระบาดขยายเป็นวงกว้างอย่างรวดเร็วไปยังประเทศต่างๆทั่วโลกรวมถึงประเทศไทยด้วย ทำให้ส่งผลกระทบต่อสาธารณสุข เศรษฐกิจ สังคม \cite{wuhan:hupei}
พยาบาลเป็นหนึ่งในทีมบุคลากรทางการแพทย์ที่มีบทบาทเป็นด่านหน้าในการควบคุมและป้องกันการแพร่ระบาดของโรค เป็นผู้ปฏิบัติงานโดยตรงกับผู้ป่วยต้องเข้าไปสัมผัสใกล้ชิดกับผู้ป่วย พยาบาลต้องตระหนักและดูแลป้องกันตนเองไม่ให้ติดเชื้อต้องมาตรฐานอย่างเคร่งครัด \cite{nurse:covid}
ทำให้พยาบาลเกิดความเหนื่อย ความเครียดจากภาระงานที่เพิ่มมากขึ้น ในการปฏิบัติงานพยาบาลต้องดูแลผู้ป่วยใน Home Isolation ที่ต้องรับคำปรึกษาตลอดเวลาและขึ้นเวร Community Isolation โรงพยาบาลสนามโดยไม่ได้หยุดพัก \cite{nurse:Isolation}
เมื่อเปรียบเทียบจำนวนพยาบาลกับสัดส่วนประชากรโดยมากถึง 1 : 353 \cite{people:nurse} ความเหนื่อยล้าจากการทํางานของพยาบาลส่งผลกระทบต่อ ความปลอดภัยของผู้ป่วยที่ลดลง การดูแลเอาใจใส่ผู้ป่วยที่อาจไม่ดีเท่าที่ควร พยาบาลต้องทํางานต่อเนื่องกันยาวนานถึง 12 ชั่วโมงอาจทําให้เกิดอัตราความผิดพลาดจากการทํางานเพิ่มขึ้น \cite{ngeaun:nurse}
โดยปกติแนวทางในการจัดตารางเวรของพยาบาลจะยึด แนวทางการบริหารการจัดตารางเวรหรือผลัด การเบิกเงินค่าตอบแทนนอกเวลา และค่าเวรหรือผลัด ของพยาบาลวิชาชีพ พยาบาลเทคนิค ผู้ช่วยพยาบาล กระทรวงสาธารณสุข เป็นหลักแต่สามารถปรับแต่งการจัดจัดตารางเวรให้เหมาะสมได้ \cite{tarang:nurse}
โดยการจัดตารางเวรหัวหน้าพยาบาลจะเป็นผู้จัดทำ การจัดตารางเวรแบบเดิมจะใช้การจดบันทึกในการดาษและเนื่องจากปัจจุบันเทคโนโลยีได้เข้ามามีบทบาทอย่างมากจึงเกิดการประยุกต์ใช้เทคโนโลยีเพื่อจัดตารางเวรของพยาบาล อาทิเช่นแอปพลิเคชัน “Microsoft Excel” ที่มาช่วยในการจัดตารางให้ดูง่ายสามารถคำนวณวันเวลาได้ลดความยุ่งยากในการจัดเก็บเอกสาร \cite{excel:headnurse}

จากการจัดตารางเวรข้างต้นของหัวหน้าพยาบาลแสดงให้เห็นว่าการจัดตารางเวรแบบเดิมหรือการจัดตารางเวรโดยการใช้แอปพลิเคชัน “Microsoft Excel” ก็ยังคงเกิดปัญหาในหลายๆเรื่อง เช่น การจัดตารางเวรจัดไม่เท่ากัน โดยอาจเกิดการเองเอียง ความเหลื่อมล้ำ โดยหัวหน้าพยาบาลไม่ได้มีข้อมูลในการชี้แจงที่ชัดเจน การจัดตารางเวรอาจไม่ได้ตรงตามความต้องการของพยาบาล ทำให้พยาบาลอาจมีการแลกเวรจำนวนมากๆ ซึ่งส่งผลให้การจัดตารางเวรนั้นเปลี่ยนไม่ตรงตามวัตถุประสงค์ที่หัวหน้าพยาบาลต้องการตั้งแต่ตั้น ข้อมูลตารางเวรหากมีการเปลี่ยนแปลงโดยการแลกเวรจะต้องทำการอัปเดตซึ่งเป็นไปได้ยาก 

จากปัญหาข้างต้นเพื่อแก้ไขปัญหาดังกล่าว ผู้จัดทำมุ่งเน้นการพัฒนาระบบผ่านเว็บแอปพลิเคชันเพื่อให้พยาบาลสามารถเข้าถึงและจัดการตารางเวรได้อย่างง่ายดายผ่านอินเทอร์เน็ต ระบบนี้จะช่วยลดความยุ่งยากในการจัดตารางเวร ตอบสนองความต้องการของพยาบาลในการจัดตารางเวร อำนวยความสะดวกในการขอลาและการแลกเวร โดยสามารถอัปเดตข้อมูลตารางเวรได้ง่ายขึ้น โดยหัวหน้าพยาบาลจะมีข้อมูลเก่าในระบบที่สามารถนำมาวิเคราะห์เพื่อจัดตารางเวรหรือตอบคำถามให้พยาบาลได้ ซึ่งทำให้ลดการถกเถียงในการจัดตารางเวร และมีภาพรวมให้ผู้อำนวยการโรงพยาบาลได้ทราบก่อนที่จะอนุมัติตารางเวรได้จึงทำให้โรงพยาบาลมีระบบระเบียบ มีประสิทธิภาพมากยิ่งขึ้น

\section{{วัตถุประสงค์}}

\hspace{0cm}\subsection{เพื่อสร้างแอปพลิเคชันช่วยเหลือหัวหน้าพยาบาลและอำนวยความสะดวกให้พยาบาล}

\hspace{0cm}\subsection{เพื่อเป็นการศึกษาและพัฒนาเว็บแอปพลิเคชัน}

\section{แนวคิดและหลักการ}

\section{ขอบเขตการศึกษา}

\section{ขั้นตอนการดำเนินงาน}

\section{แผนการดำเนินงาน}

\section{อุปกรณ์ที่ใช้ดำเนินงาน}

% \numberwithin{equation}{chapter}
\numberwithin{equation}{section}
\baselineskip=8mm
\chapter{เอกสารและงานวิจัยที่เกี่ยวข้อง}

% \renewcommand{\thesubsection}{\thechapter.\arabic{subsection}}
\renewcommand{\thesubsection}{\arabic{subsection}.}
\renewcommand{\theequation}{\thesection.\arabic{equation}}
\renewcommand{\thesection}{}



ในการศึกษา ภาคนิพนธ์เรื่อง “แอปพลิเคชันจัดตารางเวรพยาบาล” ผู้จัดทำได้ศึกษาหลักการ ทฤษฎี และงานวิจัยที่เกี่ยวข้อง เพื่อเป็นแนวทางในการศึกษาและพัฒนาภาคนิพนธ์ รายละเอียดหัวข้อดังต่อไปนี้

\subsection{แนวทางการจัดตารางเวรและการเบิกเงินค่าตอบแทนนอกเวลาของพยาบาล}

\subsection{ทฤษฎีการวิเคราะห์และออกแบบระบบ SDLC}

\subsection{ภาษาและเครื่องมือที่ใช้ในการพัฒนา}

\subsection{ทฤษฎีเกี่ยวกับความพึงพอใจ}

\subsection{งานวิจัยที่เกี่ยวข้อง}

\clearpage

\section{แนวทางการจัดตารางเวรและการเบิกเงินค่าตอบแทนนอกเวลาของพยาบาล}

อ้างถึงบันทึกของกระทรวงสาธารณะสุขที่ สธ 0202.3.7/ว 79 เรื่อง “ข้อบังคับกระทรวงสาธารณะสุขว่าด้วยการจ่ายเงินค่าตอบแทนเจ้าหน้าที่ที่ปฏิบัติงานให้กับหน่วยบริการในสังกัดกระทรวงสาธารณะสุข พ.ศ.2566 หลักเกณฑ์วิธีการและเงื่อนไขการจ่ายเงินค่าตอบแทน จำนวน 5 ฉบับ” ลงไว้เมื่อ 3 กุมภาพันธ์ 2566 ได้มีการปรับปรุงข้อบังคับฯ ดังกล่าว เพื่อให้มีความเหมาะสมกับภาวะเศรษฐกิจในปัจจุบัน และเพื่อให้เกิดความเข้าใจตรงกัน ในเรื่องการเบิกเงินค่าตอบแทนนออกเวลา (Over Time, OT) และค่าเวรหรือพลัดของพยาบาลวิชาชีพ พยาบาลเทคนิค ผู้ช่วยพยาบาล รวมทั้งให้เกิดความถูกต้องและเป็นธรรม ในเรื่องการบริหารจัดการชั่วโมงการทำงานของพยาบาล กระทรวงสาธารณสุข จึงมีหลักการและแนวทางปฎิบัติที่ผู้บริหารและผู้เกี่ยวข้องสามารถนำไปดำเนินการ ดังนี้

หลักการ


\hspace{0.5cm}\hangindent=2.7cm\hangafter=1\subsection{เวรเช้า ผลัดบ่าย ผลัดดึก (เวรพลัดๆละ 8 ชั่วโมง) เป็นการปฏิบัติงานตามปกติของพยาบาลวิชาชีพ พยาบาลเทคนิค ผู้ช่วยพยาบาล เนื่องจากเป็นผู้ที่ปฏิบัติงานผลัดเปลี่ยนหมุนเวียนกันดูแลผู้ป่วยตลอด 24 ชั่วโมง}

\hspace{0.5cm}\hangindent=2.7cm\hangafter=1\subsection{การปฏิบัติงานของพยาบาลวิชาชีพ พยาบาลเทคนิค ผู้ช่วยพยาบาล แต่ละเดือนจะมีจำนวนเวรเท่ากับวันทำการในของเดือนนั้นๆ นับว่าเป็นการปฏิบัติงานโดยปกติของพยาบาล นอกเหนือจากจำนวนเวรดังกล่าวจึงเป็นการปฏิบัติงานนอกเวลา (Over Time, OT)}

\hspace{0.5cm}\hangindent=2.7cm\hangafter=1\subsection{ในแต่ละเวรหรือผลัด ควรกำหนดให้มีพยาบาลวิชาชีพที่มีศักยภาพการปฏิบัติงานต่างระดับอย่างน้อย 2 ระดับ (Skill mix) ขึ้นไป}

\hspace{0.5cm}\hangindent=2.7cm\hangafter=1\subsection{การเบิกงานค่าตอบแทนนอกเวลา (Over Time, OT) เบิกจากจำนวนเวรหรือผลัดที่เกินจากการจัดเวรหรือผลัดปกติ โดยสามารถเบิกได้ทั้งเวรเช้า หรือผลัดบ่าย หรือผลัดดึก}

\hspace{0.4cm}\hangindent=2.7cm\hangafter=1\subsection{การเบิกเงินค่าผลัดบ่าย หรือผลัดดึก ถือเป็นการเบิกเงินในการปฏิบัติงานปกติไม่ใช่การปฏิบัติงานนอกเวลา (Over Time, OT)}


\hspace{0.5cm}\hangindent=2.7cm\hangafter=1\subsection{เวรเช้า ผลัดบ่าย ผลัดดึก สามารถจัดเวรเสริมได้ตามภาระงานที่กำหนด และเวรเสริมนั้นจะเบิกเงินค่าตอบแทนนอกเวลา (Over Time, OT)}

\section{ทฤษฎีการวิเคราะห์และออกแบบระบบ SDLC}

\section{ภาษาและเครื่องมือที่ใช้ในการพัฒนา}

\section{ทฤษฎีเกี่ยวกับความพึงพอใจ}

\section{งานวิจัยที่เกี่ยวข้อง}


% \numberwithin{equation}{chapter}
\numberwithin{equation}{section}
\baselineskip=8mm
\chapter{การวิเคราะห์และการออกแบบระบบ}

% \renewcommand{\thesubsection}{\thechapter.\arabic{subsection}}
\renewcommand{\thesubsection}{\arabic{subsection}.}
\renewcommand{\theequation}{\thesection.\arabic{equation}}
\renewcommand{\thesection}{}


\section{การออกแบบระบบ}

\section{Use Case Diagram}

\section{Use Case Description}

\section{Class Diagram}

\section{Class Description}

\section{Entity-Relationship Diagram}

\begin{figure}[h!]
   \centering
   \includegraphics[width=1\textwidth]{ER.png}
   \caption{Entity-Relationship Diagram}
\end{figure}

\section{Entity-Relationship Description}

\section{การออกแบบหน้าจอแสดงผล}

\subsection{User 1}

\subsection{User 2}

\subsection{User 3}

\subsection{User 4}

\subsection{User 5}


% \baselineskip=8mm
\numberwithin{equation}{chapter}
\numberwithin{equation}{section}
\renewcommand{\thesubsection}{\arabic{subsection}.}
\renewcommand{\theequation}{\thesection.\arabic{equation}}
\renewcommand{\thesection}{}
\renewcommand{\thesubsubsection}{\thesubsection\arabic{subsubsection}.}




\section{Use Case Diagram}

\vspace{1cm}

\begin{figure}[h]
    \centering
    \includegraphics[width=1\textwidth]{UseCase.png}
    \caption{Use Case Diagram}
    \end{figure}
\clearpage


\section{Use Case Description}

\vspace{1cm}

\begin{table}[h]
    \captionsetup{justification=raggedright,singlelinecheck=false}
    \fontsize{10}{18}\selectfont	
    \resizebox{\textwidth}{!}{%
    \centering
    \begin{tabular}{lp{10cm} l} 
        \toprule
        Use Case Title : & Manage login information  & Use Case ID : 1\\ 
        Primary Actor : & Hospital Admin , Hospital Director , Headnurse , Nurse \\ 
        Stakeholder Actor : & Application Admin \\ 
        Main Flow : & ผู้ใช้กรอก Username และ Password ในหน้าเข้าสู่ระบบ ระบบจะทำการตรวจสอบข้อมูลผู้ใช้แล้วก็จะเข้าสู่ตัวระบบและผู้ใช้สามารถออกจากระบบได้ \\ 
        Exception Flow 1 : &  กรณีที่ผู้ใช้กรอก Username หรือ Password ผิดระบบจะให้กรอกใหม่และแสดงสัญลักษณ์ว่าเข้าสู่ระบบไม่สำเร็จ \\ 
        Exception Flow 2 : &  กรณีที่ผู้ใช้ลืม Password ให้ทำการแจ้งผู้ดูแลระบบเพื่อ Reset รหัสผ่าน\\ 
        Exception Flow 3 : &  กรณีที่ผู้ใช้ไม่มีบัญชีต้องทำการติดต่อเจ้าหน้าที่ดูแลระบบ\\ \toprule    
    \end{tabular}
        \caption{Use Case 1: Manage login information}
    }
\end{table}


\begin{table}[h]
    \captionsetup{justification=raggedright,singlelinecheck=false}
    \fontsize{10}{18}\selectfont	
    \resizebox{\textwidth}{!}{%
    \centering
    \begin{tabular}{lp{10cm} l} 
        \toprule
        Use Case Title : & Ask for shifts  & Use Case ID : 2\\ 
        Primary Actor : & Nurse \\
        Stakeholder Actor : & - \\ 
        Main Flow : & พยาบาลสามารถขอเวรได้ในหน้าขอเวร พยาบาลต้องกรอกข้อมูลเวรที่ต้องการจะขอในเดือนถัดไป โดยสามารถดุการขอของพยาบาลคนอื่นได้ หลังจากกรอกข้อมูลขอเวรแล้ว ระบบจะทำการบันทึกลงฐานข้อมูล หลักการจากนั้นหากพยาบาลต้องการแก้ไขเวรที่ขอก็สามารถแก้ไขในหน้าขอเวรได้และสามารถลบเวรที่ขอมาก่อนได้ \\ 
        Exception Flow 1 : &  กรณีที่พยาบาลไม่ได้กรอกข้อมูลขอเวรในบางวัน ระบบจะทำการเซตให้เป็นค่าว่าง \\ 
        Exception Flow 2 : &  กรณีทีี่ระบบปิดการขอเวรไปแล้ว พยาบาลต้องติดต่อกับหัวหน้าพยาบาลในการจัดตารางเวรเอง\\ \toprule 
    \end{tabular}
        \caption{Use Case 2: Ask for shifts}
    }
\end{table}


\begin{table}[h]
    \captionsetup{justification=raggedright,singlelinecheck=false}
    \fontsize{12}{20}\selectfont	
    \resizebox{\textwidth}{!}{%
    \centering
    \begin{tabular}{lp{10cm} l} 
        \toprule
        Use Case Title : & Arrange a shift schedule  & Use Case ID : 3\\ 
        Primary Actor : & Headnurse \\ 
        Stakeholder Actor : & - \\ 
        Main Flow : & หัวหน้าพยาบาลสามารถจัดตารางเวรได้ โดยมีข้อมูลประกอบการตัดสินใจคือการขอเวรของพยาบาล และ ประวัติการทำงานของพยาบาลรายบุคคล หัวหน้าพยาบาลสามารถดูระดับของพยาบาลได้ หลังจากการจัดตารางเสร็จแล้วหัวหน้าพยาบาลสามารถแก้ไขตารางนั้นได้ สามารถลบได้ หากหัวหน้าพยาบาลคิดว่าเหมาะสมแล้วก็สามารถส่งเรื่องไปยังผู้อำนวยการโรงพยาบาลเพื่ออนุมัติการใช้งานตารางเวรได้ \\ 
        Exception Flow 1 : &  กรณีที่หัวหน้าพยาบาลยื่นเรื่องไปที่ผู้อำนวยการโรงพยาบาลแล้ว ต้องการแก้ไขตารางใหม่ ต้องส่งให้ผู้อำนวยการอีกรอบเพื่อยืนยัน \\ 
        Exception Flow 2 : &  กรณีที่ตารางเวรอนุมัติใช้จากผู้อำนวยการโรงพยาบาลแล้ว หากมีความประสงค์ที่จะแก้ไขตารางให้เป็นการแลกเวรแทน\\ \toprule
    \end{tabular}
        \caption{Use Case 3: Arrange a shift schedule}
    }
\end{table}


\begin{table}
    \captionsetup{justification=raggedright,singlelinecheck=false}
    \fontsize{12}{20}\selectfont	
    \resizebox{\textwidth}{!}{%
    \centering
    \begin{tabular}{lp{10cm} l} 
        \toprule
        Use Case Title : & Approve shift schedule, leave, shift exchange  & Use Case ID : 4\\ 
        Primary Actor : & Headnurse , Hospital Director \\ 
        Stakeholder Actor : & - \\ 
        Main Flow : & หัวหน้าพยาบาลและผู้อำนวยการโรงพยาบาลสามารถอนุมัติเรื่องต่างๆได้ โดยมีการกำหนดให้ผู้อำนวยการโรงพยาบาลสามารถอนุมัติตารางเวรและการลา หัวหน้าพยาบาลสามารถอนุมัติการลาและการแลกเวรของพยาบาลได้ โดยการกดยืนยันการลา หัวหน้าพยาบาลต้องยืนยันก่อนที่จะส่งเรื่องไปยังผู้อำนวยการ หากกดยืนยันแล้วจะไม่สามารถยกเลิกได้ หากไม่ยืนยันตัวระบบจะมีข้อมูลรายละเอียดให้กรอกเพื่อบอกเหตุผล \\ 
        Exception Flow 1 : &  กรณีการลาหากหัวหน้าพยาบาลยืนยันไปแล้วแต่ผู้อำนวยการโรงพยาบาลไม่ยืนยันจะถือเป็นการลาที่ไม่สำเร็จ พยาบาลต้องยื่นเรื่องเข้ามาในระบบใหม่\\ 
        Exception Flow 2 : &  กรณีที่หัวหน้าพยาบาลไม่ได้ยืนยัน ผู้อำนวยการจะไม่สามารถยืนยันได้ ต้องผ่านหัวหน้าพยาบาลก่อน\\ \toprule
    \end{tabular}
        \caption{Use Case 4: Approve shift schedule, leave, shift exchange}
    }
\end{table}


\begin{table}
    \captionsetup{justification=raggedright,singlelinecheck=false}
    \fontsize{12}{20}\selectfont	
    \resizebox{\textwidth}{!}{%
    \centering
    \begin{tabular}{lp{10cm} l} 
        \toprule
        Use Case Title : & View shift schedule  & Use Case ID : 5\\ 
        Primary Actor : & Headnurse , Nurse \\ 
        Stakeholder Actor : & - \\ 
        Main Flow : & หัวหน้าพยาบาลและพยาบาลสามารถดูตารางเวรของตัวเองได้ในตารางเวร ระบบจะแสดงข้อมูลการคำนวณต่างๆและตารางเวร หัวหน้าพยาบาลและพยาบาลสามารถเปลี่ยนรูปแบบของหน้าการแสดงผลได้ \\ 
        Exception Flow 1 : &  หากขึ้นเดือนใหม่และยังจัดตารางเวรไม่เสร็จระบบจะขึ้นตารางของเดือนก่อนให้\\ 
        \toprule
    \end{tabular}
        \caption{Use Case 5: View shift schedule}
    }
\end{table}


\begin{table}
    \captionsetup{justification=raggedright,singlelinecheck=false}
    \fontsize{12}{20}\selectfont	
    \resizebox{\textwidth}{!}{%
    \centering
    \begin{tabular}{lp{10cm} l} 
        \toprule
        Use Case Title : & Exchange shifts  & Use Case ID : 6\\ 
        Primary Actor : & Headnurse , Nurse \\ 
        Stakeholder Actor : & - \\ 
        Main Flow : & หัวหน้าพยาบาลสามารถแลกเวรกันได้หลังจากการจัดตารางเวร การแลกเวรจะแลกเวรได้โดยมีเงื่อนไขคือระดับของพยาบาลต้องมีจำนวนถึงกำหนดในวันที่แลก การแลกเวรเมื่อแลกแล้วจะต้องทำงานติดต่อกันไม่เกิน 20 ชั่วโมง  โดยสามารถแก้ไขการขอแลกเวรและลบคำขอแลกเวรได้ การขอแลกพยาบาลทั้งสองต้องสร้างการแลกเวรในระบบและกดยืนยันทั้งสอง เรื่องจะส่งไปแจ้งเตือนให้หัวหน้าพยาบาบทราบ หลังจากนั้นหัวหน้าพยาบาลจะเป็นคนตัดสินใจในการแลกเวร หากยืนยันแล้วการแลกเวรถือเป็นอันเสร็จสิ้น\\ 
        Exception Flow 1 : &  กรณีที่ต้องการแก้ไขหรือยกเลิกหลังจากการยืนยันของหัวหน้าพยาบาลจะไม่สามารถทำได้ ถ้าหากต้องการทำรายการใหม่ต้องทำการแลกเวรใหม่ตั้งแต่ต้นเท่านั้น\\ 
        \toprule
    \end{tabular}
        \caption{Use Case 6: Exchange shifts}
    }
\end{table}


\begin{table}
    \captionsetup{justification=raggedright,singlelinecheck=false}
    \fontsize{12}{20}\selectfont	
    \resizebox{\textwidth}{!}{%
    \centering
    \begin{tabular}{lp{10cm} l} 
        \toprule
        Use Case Title : & leave  & Use Case ID : 7\\ 
        Primary Actor : & Headnurse , Nurse \\ 
        Stakeholder Actor : & - \\ 
        Main Flow : & หัวหน้าพยาบาลและพยาบาลสามารถขอลาได้ หากต้องการขอลาสามารถเข้าเมนูลาและกรอกข้อมูลการลา และส่งการลาให้หัวหน้าพยาบาลยืนยันและรอให้ผู้อำนวยการยืนยัน\\ 
        Exception Flow 1 : &  กรณีที่กรอกลาแล้ว อยากยกเลิกการลาหรือแก้ไขการลาของตนเอง สามารถแก้ไขและลบได้ก่อนการยืนยันของหัวหน้าพยาบาล\\  
        \toprule
    \end{tabular}
        \caption{Use Case 7: leave}
    }
\end{table}
    

\begin{table}
    \captionsetup{justification=raggedright,singlelinecheck=false}
    \fontsize{12}{20}\selectfont	
    \resizebox{\textwidth}{!}{%
    \centering
    \begin{tabular}{lp{10cm} l} 
        \toprule
        Use Case Title : & Role management  & Use Case ID : 8\\ 
        Primary Actor : & Application Admin , Hospital Admin \\ 
        Stakeholder Actor : & - \\ 
        Main Flow : & แอดมินสามารถจัดการสิทธิ์ของผู้ใช้งานได้ โดยสามารถเพิ่มสิทธ์ ลบสิทธิ์ แก้ไขสิทธิ์และสามารถกำหนดสิทธิ์ให้ผู้ใช้ในระบบผ่านทางเมนู\\ 
        Exception Flow 1 : &  หากแอดมินลบสิทธิ์ไปแล้วผู้ใช้เป็นสิทธิ์นั้น ระบบจะกำหนดสิทธิ์เป็นค่าว่าง และแจ้งเตือนให้แอดมินทราบ \\ 
        \toprule
    \end{tabular}
        \caption{Use Case 8: Role management}
    }
\end{table}


\begin{table}
    \captionsetup{justification=raggedright,singlelinecheck=false}
    \fontsize{12}{20}\selectfont	
    \resizebox{\textwidth}{!}{%
    \centering
    \begin{tabular}{lp{10cm} l} 
        \toprule
        Use Case Title : & Manage hospital information  & Use Case ID : 9\\ 
        Primary Actor : & Application Admin , Hospital Admin \\ 
        Stakeholder Actor : & - \\ 
        Main Flow : & แอดมินสามารถจัดการข้อมูลของโรงพยาบาลโดยเข้าถึงผ่านเมนู หากต้องการเพิ่มโรงพยาบาลสามารถทำได้โดยกรอกข้อมูลของโรงพยาบาล หากทำการกรอกแล้วก็บันทึก ระบบจะบันทึกลงฐานข้อมูลและแสดงบนหน้าจอว่าบันทึกเสร็จสิ้น\\ 
        Exception Flow 1 : & กรณีที่แอดมินต้องการแก้ไขโรงพยาบาลหรือลบสามารถทำได้โดยการกดปุ่มบนเมนูของหน้าจัดการข้อมูลโรงพยาบาล \\ 
        \toprule
    \end{tabular}
        \caption{Use Case 9: Manage hospital information}
    }
\end{table}

\begin{table}
    \captionsetup{justification=raggedright,singlelinecheck=false}
    \fontsize{12}{20}\selectfont	
    \resizebox{\textwidth}{!}{%
    \centering
    \begin{tabular}{lp{10cm} l} 
        \toprule
        Use Case Title : & Set nurse level  & Use Case ID : 10\\ 
        Primary Actor : & Application Admin , Hospital Admin \\ 
        Stakeholder Actor : & - \\ 
        Main Flow : & แอดมินสามารถจัดการข้อมูลระดับของพยาบาลโดยการเข้าเมนูระดับของพยาบาล และทำการกรอกข้อมูลของระดับพยาบาล กำหนดระดับของพยาบาลกับพยาบาล \\ 
        Exception Flow 1 : &  แอดมินสามารถจัดการข้อมูลระดับของพยาบาบลโดยเข้าถึงผ่านเมนู หากต้องการเพิ่มระดับของพยาบาล สามารถทำได้โดยกรอกข้อมูลชื่อ ระดับ หากทำการกรอกแล้วก็บันทึก ระบบจะบันทึกลงฐานข้อมูลและแสดงบนหน้าจอว่าบันทึกเสร็จสิ้น \\ 
        \toprule
    \end{tabular}
        \caption{Use Case 10: Set nurse level}
    }
\end{table}
    


\begin{table}
    \captionsetup{justification=raggedright,singlelinecheck=false}
    \fontsize{12}{20}\selectfont	
    \resizebox{\textwidth}{!}{%
    \centering
    \begin{tabular}{lp{10cm} l} 
        \toprule
        Use Case Title : & View work/usage statistics  & Use Case ID : 11\\ 
        Primary Actor : & Hospital Admin , Hospital Director \\ 
        Stakeholder Actor : & - \\ 
        Main Flow : & แอดมินของโรงพยาบาลและผู้อำนวยการโรงพยาบาลสามารถดูสถิติการใช้งานของพยาบาลได้ สามารถเข้าถึผ่านทางเมนูดูสถิติการทำงาน สามารถเลือก ช่วงเวลาของวัน การทำงานของเดือน และอื่นๆเ \\ 
        Exception Flow 1 : &  กรณีที่ไม่พบข้อมูลระบบจะแสดงผลให้แอดมินหรือผู้อำนวยการทราบ \\ 
        \toprule
    \end{tabular}
        \caption{Use Case 11: View work/usage statistics}
    }
\end{table}

    
% \baselineskip=8mm
% \renewcommand{\thesubsection}{\thechapter.\arabic{subsection}}
\numberwithin{equation}{chapter}
\numberwithin{equation}{section}
\renewcommand{\thesubsection}{\arabic{subsection}.}
\renewcommand{\theequation}{\thesection.\arabic{equation}}
\renewcommand{\thesection}{}
\renewcommand{\thesubsubsection}{\thesubsection\arabic{subsubsection}.}



\section{Class Diagram}

\vspace{1cm}

\begin{figure}[h]
    \centering
    \includegraphics[width=0.63\textwidth]{Class Diagram.png}
    \caption{Class Diagram}
    \end{figure}

\clearpage




% \baselineskip=8mm
% \renewcommand{\thesubsection}{\thechapter.\arabic{subsection}}
\numberwithin{equation}{chapter}
\numberwithin{equation}{section}
\renewcommand{\thesubsection}{\arabic{subsection}.}
\renewcommand{\theequation}{\thesection.\arabic{equation}}
\renewcommand{\thesection}{}
\renewcommand{\thesubsubsection}{\thesubsection\arabic{subsubsection}.}



\section{Entity-Relationship Diagram}
\begin{figure}[h]
    \centering
    \includegraphics[width=0.9\textwidth]{ER Diagram.png}
    \caption{Entity-Relationship Diagram}
    \end{figure}

\clearpage

\section{Data Dictionary}

\vspace{1cm}


\begin{table}[h]
    \captionsetup{justification=raggedright,singlelinecheck=false}
    \centering
    \fontsize{8}{10}\selectfont	
    \resizebox{\textwidth}{!}{%
    \begin{tabular}{lccccc}
        \toprule
        \multicolumn{6}{c}{Hospitals} \\ \hline
        Name & Datatype & Length & Description & Example & Key \\ \hline
        Hospital\_id & int & 5  & รหัสโรงพยาบาล  & 00001 & PK \\
        name& char & 50 & ชื่อโรงพยาบาล & University Of Phayao Hospital &  \\
        address& Char  & 60 &  ที่อยู่โรงพยาบาล & ต.แม่กา อ.เมืองพะเยา จ.พะเยา &  \\
        phone& Char  & 10 & เบอร์โทรศัพท์ & 054466758 & \\
        description& Char & 100& ราบละเอียดของโรงพยาบาล & uph.up.ac.th & \\
        \bottomrule
    \end{tabular}%
    }
    \caption{Data Dictionary ของตาราง Hospitals}
\end{table}


\begin{table}[h]
    \captionsetup{justification=raggedright,singlelinecheck=false}
    \centering
    \fontsize{8}{10}\selectfont	
    \resizebox{\textwidth}{!}{%
    \begin{tabular}{lccccc}
        \toprule
        \multicolumn{6}{c}{Wards} \\ \hline
        Name & Datatype & Length & Description & Example & Key \\ \hline
        Ward\_id & int & 2  & รหัสวอร์ด  & 01 & PK \\
        Hospital\_id & int & 5  & รหัสโรงพยาบาล  & 00001 & FK \\
        name& char & 50 & ชื่อวอร์ด & การพยาบาลผู้ป่วยใน &  \\
        \bottomrule
    \end{tabular}%
    }
    \caption{Data Dictionary ของตาราง Wards}
\end{table}

\begin{table}[h]
    \captionsetup{justification=raggedright,singlelinecheck=false}
    \centering
    \fontsize{8}{10}\selectfont	
    \resizebox{\textwidth}{!}{%
    \begin{tabular}{lccccc}
        \toprule
        \multicolumn{6}{c}{Login} \\ \hline
        Name & Datatype & Length & Description & Example & Key \\ \hline
        Role\_id & int & 2  & รหัสสิทธิ์การใช้งาน  & 01 & FK \\
        name& char & 50 & ชื่อสิทธิ์การใช้งาน & ผู้ดูแลระบบ &  \\
        \bottomrule
    \end{tabular}%
    }
    \caption{Data Dictionary ของตาราง Role}
\end{table}

\begin{table}[h]
    \captionsetup{justification=raggedright,singlelinecheck=false}
    \centering
    \fontsize{8}{10}\selectfont	
    \resizebox{\textwidth}{!}{%
    \begin{tabular}{lccccc}
        \toprule
        \multicolumn{6}{c}{Login} \\ \hline
        Name & Datatype & Length & Description & Example & Key \\ \hline
        Login\_id & int & 5  & รหัสการเข้าสู่ระบบ  & 00001 & PK \\
        Role\_id & int & 2  & รหัสสิทธิ์การใช้งาน  & 01 & FK \\
        username& char & 30 & ชื่อผู้ใช้ & Sirawittop &  \\
        password& char & 30 & รหัสผ่าน & ZXhhbXBsZQ== &  \\
        \bottomrule
    \end{tabular}%
    }
    \caption{Data Dictionary ของตาราง Login}
\end{table}


\begin{table}[h]
    \captionsetup{justification=raggedright,singlelinecheck=false}
    \centering
    \fontsize{8}{10}\selectfont	
    \resizebox{\textwidth}{!}{%
    \begin{tabular}{lccccc}
        \toprule
        \multicolumn{6}{c}{Ots} \\ \hline
        Name & Datatype & Length & Description & Example & Key \\ \hline
        Ot\_id & int & 5  & รหัสการเข้าสู่ระบบ  & 00001 & PK \\
        Hospital\_id & int & 5  & รหัสโรงพยาบาล  & 00001 & FK \\
        Ward\_id & int & 2  & รหัสวอร์ด  & 01 & FK \\
        name& char & 20 & ชื่อตำแหน่งของพยาบาล & หัวหน้าพยาบาล &  \\
        night & int & 2 & จำนวนเวรกลางคืน & 2 &  \\
        morning & int & 2 & จำนวนเวรกลางวัน & 2 &  \\
        afternoon & int & 2 & จำนวนเวรกลางค่ำ & 2 &  \\
        datetime & datetime &  & วันที่และเวลา & 2022-12-12 12:12:12 &  \\


        \bottomrule
    \end{tabular}%
    }
    \caption{Data Dictionary ของตาราง Ots}
\end{table}

\begin{table}[h]
    \captionsetup{justification=raggedright,singlelinecheck=false}
    \centering
    \fontsize{8}{10}\selectfont	
    \resizebox{\textwidth}{!}{%
    \begin{tabular}{lccccc}
        \toprule
        \multicolumn{6}{c}{SwapShifts} \\ \hline
        Name & Datatype & Length & Description & Example & Key \\ \hline
        Swap\_id & int & 5  & รหัสการแลกเวร  & 00001 & PK \\
        User\_id & int & 5  & รหัสผู้ใช้งาน  & 00001 & FK \\
        User\_id2 & int & 5  & รหัสผู้ใช้งาน  & 00002 & FK \\
        description & char & 100 & รายละเอียดการแลกเวร & พาลูกไปแข่งขัน &  \\ 
        datetime & datetime &  & วันที่และเวลา & 2022-12-12 12:12:12 &  \\
        swap\_status & char & 10 & สถานะการแลกเวร & รอการอนุมัติ &  \\


        \bottomrule
    \end{tabular}%
    }
    \caption{Data Dictionary ของตาราง SwapShifts}
\end{table}

\begin{table}[h]
    \captionsetup{justification=raggedright,singlelinecheck=false}
    \centering
    \fontsize{8}{10}\selectfont	
    \resizebox{\textwidth}{!}{%
    \begin{tabular}{lccccc}
        \toprule
        \multicolumn{6}{c}{SwapFlow} \\ \hline
        Name & Datatype & Length & Description & Example & Key \\ \hline
        Swapflow\_id & int & 5  & รหัสความคืบหน้าการแลกเวร  & 00001 & PK \\
        Swap\_id & int & 5  & รหัสการแลกเวร  & 00001 & FK \\
        nurse\_status & char & 10 & สถานะการแลกเวรของพยาบาล & รอการอนุมัติ &  \\
        headnurse\_status & char & 10 & สถานะการแลกเวรของหัวหน้าพยาบาล & รอการอนุมัติ &  \\
        nurse\_description & char & 100 & คำอธิบายการยืนยันของพยาบาล & ไม่สะดวกในวันนี้ &  \\
        headnurse\_description & char & 100 & คำอธิบายการยืนยันของหัวหน้าพยาบาล & จำนวนพยาบาลที่มีประสบการณ์ไม่พอ &  \\
        nurse\_datetime & datetime &  & วันที่และเวลาของพยาบาลที่ยืนยัน & 2022-12-12 12:12:12 &  \\
        headnurse\_datetime & datetime &  & วันที่และเวลาของหัวหน้าพยาบาลที่ยืนยัน & 2022-12-12 12:12:12 &  \\

        \bottomrule
    \end{tabular}%
    }
    \caption{Data Dictionary ของตาราง SwapFlow}
\end{table}


\begin{table}[h]
    \captionsetup{justification=raggedright,singlelinecheck=false}
    \centering
    \fontsize{8}{10}\selectfont	
    \resizebox{\textwidth}{!}{%
    \begin{tabular}{lccccc}
        \toprule
        \multicolumn{6}{c}{Leave} \\ \hline
        Name & Datatype & Length & Description & Example & Key \\ \hline
        Leave\_id & int & 5  & รหัสการลา  & 00001 & PK \\
        User\_id & int & 5  & รหัสผู้ใช้งาน  & 00001 & FK \\
        description & char & 100 & รายละเอียดการลา & ไปงานศพ &  \\
        datetime & datetime &  & วันที่และเวลา & 2022-12-12 12:12:12 &  \\
        leave\_status & char & 10 & สถานะการลา & รอการอนุมัติ &  \\
        period & char & 10 & ช่วงเวลาการลา & ลาป่วย &  \\

        \bottomrule
    \end{tabular}%
    }
    \caption{Data Dictionary ของตาราง Leave}
\end{table}


\begin{table}[h]
    \captionsetup{justification=raggedright,singlelinecheck=false}
    \centering
    \fontsize{8}{10}\selectfont	
    \resizebox{\textwidth}{!}{%
    \begin{tabular}{lccccc}
        \toprule
        \multicolumn{6}{c}{LeaveFlow} \\ \hline
        Name & Datatype & Length & Description & Example & Key \\ \hline
        Leaveflow\_id & int & 5  & รหัสความคืบหน้าการลา  & 00001 & PK \\
        Leave\_id & int & 5  & รหัสการลา  & 00001 & FK \\
        headnurse\_status & char & 10 & สถานะการยืนยันของหัวหน้าพยาบาล & รอการอนุมัติ &  \\   
        headnurse\_description & char & 100 & คำอธิบายการยืนยันของหัวหน้าพยาบาล & จำนวนพยาบาลที่มีประสบการณ์ไม่พอ &  \\
        headnurse\_datetime & datetime &  & วันที่และเวลาของหัวหน้าพยาบาลที่ยืนยัน & 2022-12-12 12:12:12 &  \\
        director\_status     & char & 10 & สถานะการยืนยันของผู้อำนวยการ & รอการอนุมัติ &  \\
        director\_description & char & 100 & คำอธิบายการยืนยันของผู้อำนวยการ & ไม่สามารถลาได้เพราะลาบ่อย &  \\
        director\_datetime    & datetime &  & วันที่และเวลาของผู้อำนวยการที่ยืนยัน & 2022-12-12 12:12:12 &  \\


        \bottomrule
    \end{tabular}%
    }
    \caption{Data Dictionary ของตาราง LeaveFlow}
\end{table}


\begin{table}[h]
    \captionsetup{justification=raggedright,singlelinecheck=false}
    \centering
    \fontsize{8}{10}\selectfont	
    \resizebox{\textwidth}{!}{%
    \begin{tabular}{lccccc}
        \toprule
        \multicolumn{6}{c}{Users} \\ \hline
        Name & Datatype & Length & Description & Example & Key \\ \hline
        User\_id & int & 5  & รหัสผู้ใช้งาน  & 00001 & PK \\
        Hospital\_id & int & 5  & รหัสโรงพยาบาล  & 00001 & FK \\
        Ward\_id & int & 2  & รหัสวอร์ด  & 01 & FK \\
        Login\_id & int & 5  & รหัสการเข้าสู่ระบบ  & 00001 & FK \\
        Nurselevel\_id & int & 2  & รหัสระดับของพยาบาล  & 01 & FK \\  
        firstname & char & 30 & ชื่อ & Sirawit &  \\
        lastname & char & 30 & นามสกุล & Opas &  \\
        gender & char & 10 & เพศ & ชาย &  \\
        position & char & 30 & ตำแหน่ง & พยาบาล &  \\
        phone\_number & char & 10 & เบอร์โทรศัพท์ & 0987654321 &  \\
        email & char & 50 & อีเมล & sirawit.code@gmail.com & \\
        employment\_year & int & 4 & ปีการทำงาน & 5 &  \\            
        \bottomrule
    \end{tabular}%
    }
    \caption{Data Dictionary ของตาราง Users}
\end{table}


\begin{table}[h]
    \captionsetup{justification=raggedright,singlelinecheck=false}
    \centering
    \fontsize{8}{10}\selectfont	
    \resizebox{\textwidth}{!}{%
    \begin{tabular}{lccccc}
        \toprule
        \multicolumn{6}{c}{LeaveType} \\ \hline
        Name & Datatype & Length & Description & Example & Key \\ \hline
        LeaveType\_id & int & 5  & รหัสประเภทการลา  & 00001 & PK \\
        typename & char & 30 & ชื่อประเภทการลา & ลาป่วย &  \\
        \bottomrule
    \end{tabular}%
    }
    \caption{Data Dictionary ของตาราง LeaveType}
\end{table}

\begin{table}[h]
    \captionsetup{justification=raggedright,singlelinecheck=false}
    \centering
    \fontsize{8}{10}\selectfont	
    \resizebox{\textwidth}{!}{%
    \begin{tabular}{lccccc}
        \toprule
        \multicolumn{6}{c}{NurseLevel} \\ \hline
        Name & Datatype & Length & Description & Example & Key \\ \hline
        Nurselevel\_id & int & 2  & รหัสระดับของพยาบาล  & 01 & PK \\
        name & char & 30 & ชื่อระดับของพยาบาล & พยาบาลปฏิบัติการ &  \\
        \bottomrule
    \end{tabular}%
    }
    \caption{Data Dictionary ของตาราง NurseLevel}
\end{table}


\begin{table}[h]
    \captionsetup{justification=raggedright,singlelinecheck=false}
    \centering
    \fontsize{8}{10}\selectfont	
    \resizebox{\textwidth}{!}{%
    \begin{tabular}{lccccc}
        \toprule
        \multicolumn{6}{c}{Plantypes} \\ \hline
        Name & Datatype & Length & Description & Example & Key \\ \hline
        Plantypes\_id & int & 2  & รหัสแผนการทำงาน  & 01 & PK \\
        typename & char & 30 & ชื่อแผนการทำงาน & ลาป่าย &  \\
        night & boolean & 2 & เวรดึก & 1 &  \\
        morning & boolean & 2 & เวรเช้า & 1 &  \\
        afternoon & boolean & 2 & เวรบ่าย & 1 &  \\
        x & boolean & 2 & เวรหยุด & 0 &  \\
        v & boolean & 2 & ลาพักผ่อน & 1 &  \\
        n & boolean & 2 & นอกเวลา & 1 &  \\
        c & boolean & 2 & on call duty & 1 &  \\
        otn & boolean & 2 & การทำงานนอกเวลาดึก & 1 &  \\
        otm & boolean & 2 & การทำงานนอกเวลาเช้า & 0 &  \\
        ota & boolean & 2 & การทำงานนอกเวลาบ่าย & 0 &  \\
        w8 & boolean & 2 & ทำงานแบบ 8 ชั่วโมง & 0 &  \\
        w12 & boolean & 2 & ทำงานแบบ 12 ชั่วโมง & 1 &  \\
        \bottomrule
    \end{tabular}%
    }
    \caption{Data Dictionary ของตาราง Plantypes}
\end{table}


\begin{table}[h]
    \captionsetup{justification=raggedright,singlelinecheck=false}
    \centering
    \fontsize{8}{10}\selectfont	
    \resizebox{\textwidth}{!}{%
    \begin{tabular}{lccccc}
        \toprule
        \multicolumn{6}{c}{Plans} \\ \hline
        Name & Datatype & Length & Description & Example & Key \\ \hline
        Plans\_id & int & 5  & รหัสแผนการทำงาน  & 00001 & PK \\
        Hospital\_id & int & 5  & รหัสโรงพยาบาล  & 00001 & FK \\
        Ward\_id & int & 2  & รหัสวอร์ด  & 01 & FK \\
        User\_id & int & 5  & รหัสผู้ใช้งาน  & 00001 & FK \\
        Ots\_id & int & 5  & รหัสการทำงานนอกเวลา  & 00001 & FK \\
        datetime & datetime &  & วันที่และเวลา & 2022-12-12 12:12:12 &  \\
        day1 & int & 3 & วันที่1  & 1 &  \\
        day2 & int & 3 & วันที่2  & 4 &  \\
        ... & ... & ... & ... & ... &  \\
        day31 & int & 3 & วันที่31  & 6 &  \\


        \bottomrule
    \end{tabular}%
    }
    \caption{Data Dictionary ของตาราง Plans}
\end{table}
% \baselineskip=8mm
% \renewcommand{\thesubsection}{\thechapter.\arabic{subsection}}
\numberwithin{equation}{chapter}
\numberwithin{equation}{section}
\renewcommand{\thesubsection}{\arabic{subsection}.}
\renewcommand{\theequation}{\thesection.\arabic{equation}}
\renewcommand{\thesection}{}
\renewcommand{\thesubsubsection}{\thesubsection\arabic{subsubsection}.}


\section{Sequence Diagram}

\vspace{1cm}



\begin{figure}[h]
    \centering
    \includegraphics[width=0.7\textwidth]{Sequence 1.1.png}
    \caption{Sequence Diagram Login}
    \end{figure}

% sequence diagram 1.2
\begin{figure}[h]
    \centering
    \includegraphics[width=0.7\textwidth]{Sequence 1.2.png}
    \caption{Sequence Diagram Edit Username}
    \end{figure}

\begin{figure}[h]
    \centering
    \includegraphics[width=0.8\textwidth]{Sequence 1.2.1.png}
    \caption{Sequence Diagram Edit Password}
\end{figure}

% sequence diagram 1.3
\begin{figure}[h]
    \centering
    \includegraphics[width=0.8\textwidth]{Sequence 1.3.png}
    \caption{Sequence Diagram Logout}
    \end{figure}



    \begin{figure}[h]
    \centering
    \includegraphics[width=0.7\textwidth]{Sequence 2.1.png}
    \caption{Sequence Diagram AddShifts}
    \end{figure}

    \begin{figure}[h]
    \centering
    \includegraphics[width=0.7\textwidth]{Sequence 2.2.png}
    \caption{Sequence Diagram EditShifts}
    \end{figure}

    \begin{figure}[h]
    \centering
    \includegraphics[width=0.7\textwidth]{Sequence 2.3.png}
    \caption{Sequence Diagram DeleteShifts}
    \end{figure}

    \begin{figure}[h]
    \centering
    \includegraphics[width=0.8\textwidth]{Sequence 2.4.png}
    \caption{Sequence Diagram ViewFriendShifts}
    \end{figure}




    \begin{figure}[h]
    \centering
    \includegraphics[width=0.6\textwidth]{Sequence 3.1.png}
    \caption{Sequence Diagram AddSchedule}
    \end{figure}

    \begin{figure}[h]
    \centering
    \includegraphics[width=0.6\textwidth]{Sequence 3.2.png}
    \caption{Sequence Diagram EditSchedule}
    \end{figure}

    \begin{figure}[h]
    \centering
    \includegraphics[width=0.6\textwidth]{Sequence 3.3.png}
    \caption{Sequence Diagram DeleteSchedule}
    \end{figure}

    \begin{figure}[h]
    \centering
    \includegraphics[width=0.6\textwidth]{Sequence 3.4.png}
    \caption{Sequence Diagram ExportSchedule}
    \end{figure}



    \begin{figure}[h]
    \centering
    \includegraphics[width=0.6\textwidth]{Sequence 4.1.png}
    \caption{Sequence Diagram ApproveExchange}
    \end{figure}

    \begin{figure}[h]
    \centering
    \includegraphics[width=0.6\textwidth]{Sequence 4.2.png}
    \caption{Sequence Diagram ApproveLeave}
    \end{figure}

    \begin{figure}[h]
    \centering
    \includegraphics[width=0.6\textwidth]{Sequence 4.3.png}
    \caption{Sequence Diagram ApproveSchedule}
    \end{figure}



    \begin{figure}
    \centering
    \includegraphics[width=0.6\textwidth]{Sequence 4.4.png}
    \caption{Sequence Diagram ApprovePlan}
    \end{figure}


    \begin{figure}[h]
    \centering
    \includegraphics[width=0.6\textwidth]{Sequence 5.1.png}
    \caption{Sequence Diagram ViewSchedule}
    \end{figure}


    \begin{figure}[h]
    \centering
    \includegraphics[width=0.6\textwidth]{Sequence 6.1.png}
    \caption{Sequence Diagram ExchangeShifts}
    \end{figure}
    
    \begin{figure}[h]
        \centering
        \includegraphics[width=0.6\textwidth]{Sequence 6.2.png}
        \caption{Sequence Diagram EditExchange}
        \end{figure}

    \begin{figure}[h]
    \centering
    \includegraphics[width=0.6\textwidth]{Sequence 6.3.png}
    \caption{Sequence Diagram DeleteExchange}
    \end{figure}


    \begin{figure}[h]
        \centering
        \includegraphics[width=0.6\textwidth]{Sequence 6.4.png}
        \caption{Sequence Diagram HistoryExchange}
        \end{figure}
    
        \begin{figure}[h]
            \centering
            \includegraphics[width=0.6\textwidth]{Sequence 6.5.png}
            \caption{Sequence Diagram ProgresExchange}
            \end{figure}


\begin{figure}[h]
    \centering
    \includegraphics[width=0.6\textwidth]{Sequence 7.1.png}
    \caption{Sequence Diagram AddLeave}
    \end{figure}


    \begin{figure}[h]
    \centering
    \includegraphics[width=0.6\textwidth]{Sequence 7.2.png}
    \caption{Sequence Diagram EditLeave}
    \end{figure}

    \begin{figure}[h]
    \centering
    \includegraphics[width=0.6\textwidth]{Sequence 7.3.png}
    \caption{Sequence Diagram DeleteLeave}
    \end{figure}

    \begin{figure}[h]
    \centering
    \includegraphics[width=0.6\textwidth]{Sequence 7.4.png}
    \caption{Sequence Diagram HistoryLeave}
    \end{figure}


    \begin{figure}[h]
    \centering
    \includegraphics[width=0.6\textwidth]{Sequence 7.5.png}
    \caption{Sequence Diagram ProgresLeave}
    \end{figure}



\begin{figure}[h]
    \centering
    \includegraphics[width=0.6\textwidth]{Sequence 8.1.png}
    \caption{Sequence Diagram ViewRole}
    \end{figure}

    \begin{figure}[h]
    \centering
    \includegraphics[width=0.6\textwidth]{Sequence 8.2.png}
    \caption{Sequence Diagram AddRole}
    \end{figure}

    \begin{figure}[h]
    \centering
    \includegraphics[width=0.6\textwidth]{Sequence 8.3.png}
    \caption{Sequence Diagram ChangeRole}
    \end{figure}

    \begin{figure}[h]
    \centering
    \includegraphics[width=0.6\textwidth]{Sequence 8.4.png}
    \caption{Sequence Diagram editRole}
    \end{figure}

    \begin{figure}[h]
    \centering
    \includegraphics[width=0.6\textwidth]{Sequence 8.5.png}
    \caption{Sequence Diagram DeleteRole}
    \end{figure}


\begin{figure}[h]
    \centering
    \includegraphics[width=0.6\textwidth]{Sequence 9.1.png}
    \caption{Sequence Diagram AddHospital}
    \end{figure}

    \begin{figure}[h]
    \centering
    \includegraphics[width=0.6\textwidth]{Sequence 9.2.png}
    \caption{Sequence Diagram EditHospital}
    \end{figure}

    \begin{figure}[h]
    \centering
    \includegraphics[width=0.6\textwidth]{Sequence 9.3.png}
    \caption{Sequence Diagram DeleteHospital}
    \end{figure}

    \begin{figure}[h]
    \centering
    \includegraphics[width=0.6\textwidth]{Sequence 9.4.png}
    \caption{Sequence Diagram AddHospitalAdmin}
    \end{figure}

    \begin{figure}[h]
    \centering
    \includegraphics[width=0.6\textwidth]{Sequence 9.5.png}
    \caption{Sequence Diagram DeleteHospitalAdmin}
    \end{figure}


\begin{figure}[h]
    \centering
    \includegraphics[width=0.6\textwidth]{Sequence 10.1.png}
    \caption{Sequence Diagram AddNurseLevel}
    \end{figure}

    \begin{figure}[h]
    \centering
    \includegraphics[width=0.6\textwidth]{Sequence 10.2.png}
    \caption{Sequence Diagram SetNurseLevel}
    \end{figure}

    \begin{figure}[h]
    \centering
    \includegraphics[width=0.6\textwidth]{Sequence 10.3.png}
    \caption{Sequence Diagram EditNurseLevel}
    \end{figure}

    \begin{figure}[h]
    \centering
    \includegraphics[width=0.6\textwidth]{Sequence 10.4.png}
    \caption{Sequence Diagram DeleteNurseLevel}
    \end{figure}


\begin{figure}[h]
    \centering
    \includegraphics[width=0.6\textwidth]{Sequence 11.1.png}
    \caption{Sequence Diagram ViewMorningStatistics}
    \end{figure}

    \begin{figure}[h]
    \centering
    \includegraphics[width=0.6\textwidth]{Sequence 11.2.png}
    \caption{Sequence Diagram ViewAfternoonStatistics}
    \end{figure}

    \begin{figure}[h]
    \centering
    \includegraphics[width=0.6\textwidth]{Sequence 11.3.png}
    \caption{Sequence Diagram ViewNightStatistics}
    \end{figure}

    \begin{figure}[h]
    \centering
    \includegraphics[width=0.6\textwidth]{Sequence 11.4.png}
    \caption{Sequence Diagram ViewOTStatistics}
    \end{figure}

    \begin{figure}[h]
    \centering
    \includegraphics[width=0.6\textwidth]{Sequence 11.5.png}
    \caption{Sequence Diagram ViewExchangeStatistics}
    \end{figure}

    \begin{figure}[h]
    \centering
    \includegraphics[width=0.6\textwidth]{Sequence 11.6.png}
    \caption{Sequence Diagram ViewNurseStatistics}
    \end{figure}


\baselineskip=8mm
% \renewcommand{\thesubsection}{\thechapter.\arabic{subsection}}
\numberwithin{equation}{chapter}
\numberwithin{equation}{section}
\renewcommand{\thesubsection}{\arabic{subsection}.}
\renewcommand{\theequation}{\thesection.\arabic{equation}}
\renewcommand{\thesection}{}
\renewcommand{\thesubsubsection}{\thesubsection\arabic{subsubsection}.}




\section{การออกแบบส่วนเชื่อมต่อประสานกับผู้ใช้}

ในการออกแบบส่วนต่อประสานกับผู้ใช้ของแอปพลิเคชันจัดตารางเวรพยาบาลจะเน้นไปที่ความเรียบง่าย สบายตา และใช้งานได้ง่าย เพื่อให้บุคคลากรณ์ทางการแพทย์ใช้ได้ง่ายและรวดเร็ว โดยจะมีการออกแบบหน้าจอต่างๆ แบ่งตามประเภทของผู้ใช้ได้ดังนี้


\subsection{Application Admin}

\begin{figure}[h]
    \centering
    \includegraphics[width=0.6\textwidth]{Login ui.png}
    \caption{หน้าต่างการเข้าสู่ระบบของแอดมินแอปพลิเคชัน}
    \end{figure}

\subsection{Hospital Admin}

\begin{figure}[h]
    \centering
    \includegraphics[width=0.6\textwidth]{Login ui.png}
    \caption{หน้าต่างการเข้าสู่ระบบของแอดมินโรงพยาบาล}
    \end{figure}

\subsection{Hospital Director}

\begin{figure}[h]
    \centering
    \includegraphics[width=0.6\textwidth]{Login ui.png}
    \caption{หน้าต่างการเข้าสู่ระบบของผู้อำนวยการโรงพยาบาล}
    \end{figure}

\subsection{Headnurse}

\begin{figure}[h]
    \centering
    \includegraphics[width=0.6\textwidth]{Login ui.png}
    \caption{หน้าต่างการเข้าสู่ระบบของหัวหน้าพยาบาล}
    \end{figure}

\subsection{Nurse}

\begin{figure}[h]
    \centering
    \includegraphics[width=0.6\textwidth]{Login ui.png}
    \caption{หน้าต่างการเข้าสู่ระบบของพยาบาล}
    \end{figure}

\begin{figure}
    \centering
    \includegraphics[width=0.6\textwidth]{Home ui.png}
    \caption{หน้าต่างหลักของพยาบาล}
\end{figure}

%\include{chap5}
%\include{chap6}
%\include{chap7}
%\include{chap8}
%\include{chap9}
% การเขียนเอกสารอ้างอิง ให้ใช้รูปแบบ APA7 
% ลักษณะของการเขียนนี้ เป็นแบบ bibtex หากใช้รูปแบบอื่น
% ให้ปรับเป็นแบบ APA7 
%
% หนังสือทั่วไป ให้ใช้รูปแบบ โดยชื่อเรื่องต้องเป็นตัวเอียง ผู้แต่ง 1 - 20 คน ให้ใส่ชื่อทุกคน
%
% ผู้แต่ง. (ปีที่พิมพ์). \textit{ชื่อเรื่อง} (ครั้งที่พิมพ์ พิมพ์ครั้งที่ 2 เป็นต้นไป). สำนักพิมพ์.
%
% เช่น ภาสกร เนตรทิพย์วัลย์, พรพรรณ ภูสาหัส, และวิถี ธุระธรรม. (2565). \textit{การตรวจร่างกาย} (พิมพ์ครั้งที่ 2). พิมพ์ดีการพิมพ์.
% หรือ Kee, J. L., Marshall, S. M., \& Forrester, M. C. (2021). \textit{Clinical calculations} (9th ed.). Elsevier.
%
% ปล. ถ้าไม่ปรากฏปีที่พิมพ์ให้ใส่ (ม.ป.ป.) สำหรับภาษาไทย และ (n.d.) สำหรับภาษาอังกฤษ
%
% วารสารงานวิจัย ให้ใช้รูปแบบดังนี้
%
% ชื่อผู้เขียนบทความ. (ปีที่พิมพ์). ชื่อบทความ. \textit{ชื่อวารสาร}, \textit{เลขฉบับที่ หรือ Volume}(ฉบับที่ หรือ issue), หน้าแรก-หน้าสุดท้าย.
%
% เช่น บุศรา ชัยทัศน์. (2559). การดูแลผู้ป่วยโรงมะเร็งลำไส้ใหญ่. \textit{วารสารพยาบาลสภากาชาดไทย}, \textit{9}(1),19-33.
%หรือ Plows, J. F., Stanley, J. L., \& Vickers, M. H. (2018). The pathophysiology of gestational diabetes metllitus. \textit{International journal of molecular sciences}, \textit{19}(11), 3342.
% https:/doi.org/10.3390/ijms19113342 [arXiv:2210.07273 [astro-ph.HE]]
%
% ปล. ต้องใส่เลข doi ในรูปแบบลิงค์ https และอ้างอิง arXiv ใน [...] ด้วย ถ้ามี
%
% ดูเพิ่มเติมจาก https://tinyurl.com/4p6c5mf5

\begin{thebibliography}{99}

\bibitem{wuhan:hupei}
    กรมควบคุมโรค. (2563). \textbf{คู่มือเจ้าหน้าที่สาธารณะสุขในการโต้ตอบภาวะฉุกเฉิน กรณีการระบาดโรคติดเชื้อไวรัสโคโรนา 2019 ในประเทศไทย.} ม.ป.ท.: ม.ป.พ.
\bibitem{nurse:covid}
    นราจันทร์ ปัญญาวุทโส, ปรัชญานันท์ เที่ยงจรรยา และประภาพร ชูกำเหนิด. (2565). \\วารสารมหาวิทยาลัยคริสเตียน. \textbf{ประสบการณ์ของพยาบาลวิชาชีพในการมีส่วนร่วม\\ด้านความปลอดภัยในภาวะวิกฤตของการแพร่ระบาดโรคโควิด 19 โรงพยาบาลหาดใหญ่ ประเทศไทย,} 28, 59-72.
\bibitem{nurse:Isolation}
    คณะกรรมาธิการการสาธารณะสุข วุฒิสภา. (2565). \textbf{ภาระงานและประสิทธิภาพของวิชาชีพพยาบาล ภายใต้สถานะการณ์การระบาดของโรค COVID 19.} ม.ป.ท.: ม.ป.พ.
\bibitem{people:nurse}
    สำนักงานปลัดกระทรวงสาธารณะสุข กระทรวงสาธารณะสุข. (2564). \textbf{สัดส่วนเจ้าหน้าที่ทางการแพทย์ต่อประชากร.} ม.ป.ท.: ม.ป.พ.
\bibitem{ngeaun:nurse}
    เกศินี กิตติบาล, อารี ชีวเกษมสุข และชูชาติ พ่วงสมจิตร์. (2564). วารสารพยาบาลโรคหัวใจและทรวงอก. \textbf{การจัดการความเหนื่อยล้าจากการทํางานของพยาบาลวิชาชีพ \\โรงพยาบาลพระนครศรีอยุธยา,} 32, 121-136.
\bibitem{tarang:nurse}
    กองการพยาบาล กระทรวงสาธารณะสุข. (2566). \textbf{แนวทางการบริหารการจัดตารางเวรหรือผลัด \\การเบิกเงินค่าตอบแทนนอกเวลาและค่าเวรหรือผลัดของพยาบาลวิชาชีพ พยาบาลเทคนิค ผู้ช่วยพยาบาล กระทรวงสาธารณสุข.} ม.ป.ท.: ม.ป.พ.
\bibitem{excel:headnurse}
    ปริวัฒณ์ อารีชาติ และคณะ. (2565). \textbf{Thai Journal of Operations Research: TJOR. ตัวแบบการจัดตารางเวรของเภสัชกรเพื่อลดความเหลื่อมล้ําของภาระงาน,} 10, 103-112
\end{thebibliography}



%\backmatter%%%%%%%%%%%%%%%%%%%%%%%%%%%%%%%%%%%%%%%%%%%%%%%%%%%%%%%

%\renewcommand{\bibname}{บรรณานุกรม}
%\bibliographystyle{myref}
%\nocite{*}
%\bibliography{References}


%%%%%%%%%%%%%%%%%%%%%%%%%%%%%%%%%%%%%%%%%
%%%%%%%%%%%%%%%%%%%%%%%%%%%%%%%%%%%%%%%%%
% CREATE THE BIBLIOGRAPHY  สร้างหน้าประวัติผู้เขียน
%%%%%%%%%%%%%%%%%%%%%%%%%%%%%%%%%%%%%%%%%
%%%%%%%%%%%%%%%%%%%%%%%%%%%%%%%%%%%%%%%%%
\clearpage

% not show ประวัติผู้วิจัย
% \renewcommand{\arraystretch}{1}
% \makebiography                          % Generate your biography page
% %\makeWorkExpr                           % Generate your work experiences
% \makeEduBG
%\makePublication
%\makePresentation

\end{document}

